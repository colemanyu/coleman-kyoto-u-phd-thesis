\section{Experiments} \label{sec:experiment}

We evaluate the performance of our proposed distance measure framework, PSD, via its two instantiations: PSED and PSDTW. 
Specifically, we focus on a query retrieval task in which the query exhibits piecewise-scaled distortion, that is, distinct phases of the query exhibit different expression rates relative to the target in the candidate dataset. 
Our objective is to verify whether PSD achieves invariance to these multiple scaling distortions, thereby allowing it to correctly retrieve the most similar candidate. 
We detail the experimental setup in Section~\ref{sec:experimental-setup} and present the results in Section~\ref{sec:experimental-results}. The code and data are available at \url{https://github.com/colemanyu/k-scaling-factor-dtw}.

\subsection{Experimental Setup} \label{sec:experimental-setup}

We utilize the GunPoint dataset, originally released in 2003~\cite{ratanamahatana2004everything}, as the running example. 
Widely regarded as the ``iris'' dataset of the time series community~\cite{dau2019ucra}, it has appeared in over one thousand publications. 
Beyond its ubiquity, the dataset addresses a critical distinction: misinterpreting the act of aiming a gun as merely pointing a finger could have life-threatening consequences.

\begin{figure}[htbp]
\centerline{\includegraphics[width=1.0\columnwidth]{figures/GunPoint.pptx.pdf}}
\caption{Visualization of the GunPoint dataset.
Left (Right): A time series of the ``Gun'' (``Point'') scenario. 
Critical periods, such as ``Aiming'', are annotated.}
\label{fig:GunPoint}
\end{figure} \begin{figure}[htbp]
\centerline{\includegraphics[width=0.6\columnwidth]{figures/concatenation_uniform_vs_random.pdf}}
\caption{The red solid time series shows an instant in the target set. The blue dashed time series shows an instant in the query set.}
\label{fig:us-concatenation_uniform_vs_random}
\end{figure}

The dataset contains two classes: ``Gun'' and Point''. Actors aim at an eye-level target using either a replica gun or their index finger, as illustrated in Figure~\ref{fig:GunPoint}. 
The resulting time series represent the X-axis centroid of the actor's right hand. 
Each action lasts for 5 seconds, with the pointing/aiming phase occurring for approximately one second. 
Recorded at 30 fps, each sample consists of 150 data points. The dataset comprises 50 training and 150 test series, all of length 150.

A key limitation of the original dataset is the assumption that every action lasts exactly 5 seconds. 
In reality, actors perform actions at varying speeds. 
If an actor is asked to perform the action three times continuously in a row, we are likely to observe a time series containing three phases, each with a unique rate. 
The first phase should take longer than subsequent phases because it is the first time the action is performed.
This phenomenon is depicted in Figure~\ref{fig:us-concatenation_uniform_vs_random}, where the red curve represents an ideal case that consists of three identical phases (i.e., identical rate), while the blue curve represents a realistic scenario with varying rates.
The first action is slower than the second action.
Consequently, retrieving such patterns requires assigning different scaling factors to each phase to accommodate the phase-specific rates.

We now describe the procedure for generating the target and query sets for the retrieval task. 
To construct the target set, we concatenate $P$ repetitions of each time series instance from the source dataset. 
To ensure a fair comparison, we constrain the resulting time series to match the original length $n$. 
This is achieved by rescaling each phase to a length of $n/P$ prior to concatenation. 
Note that we must handle remainders to ensure that the final time series length is exactly $n$.
An example of such a target (where $P=3$) is depicted by the red solid line in Figure~\ref{fig:us-concatenation_uniform_vs_random}.

To construct the query set, we first determine the specific lengths for the $P$ segments. Starting with an expected mean length of $n/P$, we define the minimum segment length as $L_{\min} = (n/P)\sqrt{l}$ and the maximum as $L_{\max} = (n/P)\sqrt{l}$. 
This formulation ensures that the ratio of the lengths of any two segments is bounded by $l$. 
We then generate $P$ random integers within $[L_{\min}, L_{\max}]$ subject to the constraint that their sum is exactly $n$. Finally, we construct the query by rescaling $P$ copies of the source time series to these generated random lengths and concatenating them. The resulting time series maintains the original length $n$. An example of such a query (where $P=3$) is depicted by the blue dashed line in Figure~\ref{fig:us-concatenation_uniform_vs_random}.

Both the query and target sets are derived from the same source dataset. Consequently, the ground truth target for a given query is defined as the instance in the target set that is generated from the same underlying source time series.

\begin{table}[ht]
\centering
\caption{Details of the ten datasets for the experiments}
\resizebox{\columnwidth}{!}{
\begin{tabular}{lccccc}
\toprule
Name & Type & Train/Test? & Size & Class & Length \\
\midrule
SonyAIBORobotSurface1 & Sensor & Test & 601 & 2 & 70 \\
ECG200 & ECG & Train & 100 & 2 & 96 \\
MedicalImages & Image & Train & 381 & 10 & 99 \\
CBF & Simulated & Train & 30 & 3 & 128 \\
SwedishLeaf & HAR & Train & 500 & 15 & 128 \\
Plane & Sensor & Train & 105 & 7 & 144 \\
PowerCons & Device & Train & 180 & 2 & 144 \\
GunPoint & HAR & Train & 50 & 2 & 150 \\
Adiac & Image & Train & 390 & 37 & 176 \\
Epilepsy & HAR & Train & 137 & 4 & 207 \\
\bottomrule
\end{tabular}
}
\label{tab:datasets_details}
\end{table}
Table~\ref{tab:datasets_details} details the additional datasets used in this study. The column labeled ``Train/Test?'' specifies which split was employed as the source dataset.

In our experiments, we set the warping window parameter $r = 0.1$, as suggested in the literature.
The algorithms were implemented in Python.
We utilized the \texttt{aeon}~\cite{middlehurst2024aeon} library to obtain the baseline distance measures. 
All experiments were conducted on a workstation equipped with an Intel Xeon Gold 6326 CPU and 256GB of RAM.

\subsection{Experimental Results} \label{sec:experimental-results}
We employ Top-$k$ Accuracy (often referred to as Precision@k~\cite{tatbul2018precision}) as the primary evaluation metric. 
For a single query $Q$, this metric indicates whether the correct match is successfully retrieved within the top $k$ candidates:
\begin{equation} 
\label{eq:precision-k}
P@k(Q) = 
\begin{cases}
    1 & \text{if the ground truth is in the top-$k$ results}\\
    0 & \text{otherwise}\\
\end{cases}
\end{equation}

To determine the top-$k$ results, we compute the distance between $Q$ and every time series in the target set, generating a distance profile of length equal to the dataset size. 
The top-$k$ results correspond to the $k$ instances with the smallest distances in this profile. 
Finally, we evaluate the overall performance by computing the mean Top-$k$ Accuracy across the entire query set $\mathcal{D}$:
\begin{equation} 
\label{eq:precision-k-mean}
\overbar{P@k} = \frac{\sum_{Q \in \mathcal{D}}{P@k(Q)}}{|\mathcal{D}|}
\end{equation}

We choose $k \in {1, 3}$. 
$k=1$ refers to the exact retrieval. 
$k=3$ give some tolerance for the retrieval. 

\begin{table*}[ht]
\centering
\caption{The accuracy comparison for eight distance measures of the GunPoint dataset}
\setlength{\tabcolsep}{4pt} % (Default is 6pt.)
\resizebox{1.0\columnwidth}{!}{
\begin{tabular}{cc|cc|cccccccccccc|cccc}
\toprule
\multirow{2}{*}{$P$} & \multirow{2}{*}{$l$} 
& \multicolumn{2}{c|}{ED}
& \multicolumn{2}{c}{DTW~\cite{sakoe1978dynamic}} 
& \multicolumn{2}{c}{ADTW~\cite{herrmann2023amercing}} 
& \multicolumn{2}{c}{DDTW~\cite{keogh2001derivative}} 
& \multicolumn{2}{c}{shapeDTW~\cite{zhao2018shapedtw}} 
& \multicolumn{2}{c}{WDDTW~\cite{jeong2011weighted}} 
& \multicolumn{2}{c}{WDTW~\cite{jeong2011weighted}} 
& \multicolumn{2}{|c}{PSED} 
& \multicolumn{2}{c}{PSDTW} \\
% \cmidrule(lr){3-4} \cmidrule(lr){5-6}
& & 
$\overline{P@1}$ & $\overline{P@3}$ &
$\overline{P@1}$ & $\overline{P@3}$ & 
$\overline{P@1}$ & $\overline{P@3}$ & 
$\overline{P@1}$ & $\overline{P@3}$ & 
$\overline{P@1}$ & $\overline{P@3}$ & 
$\overline{P@1}$ & $\overline{P@3}$ & 
$\overline{P@1}$ & $\overline{P@3}$ & 
$\overline{P@1}$ & $\overline{P@3}$ & 
$\overline{P@1}$ & $\overline{P@3}$ \\
\midrule
\multirow{4}{*}{2} 
& 1.25 & 0.30 & 0.56 & 0.82 & 1.00 & 0.54 & 0.84 & 0.84 & 0.92 & \underline{0.96} & 1.00 & 0.88 & 0.96 & 0.82 & 0.98 & \textbf{0.98} & 1.00 & 0.86 & 1.00 \\
& 1.50 & 0.14 & 0.36 & 0.88 & 1.00 & 0.38 & 0.64 & 0.78 & 0.94 & \textbf{1.00} & 1.00 & 0.80 & 0.92 & 0.88 & 0.96 & \textbf{1.00} & 1.00 & \underline{0.96} & 1.00 \\
& 1.75 & 0.16 & 0.34 & 0.82 & 1.00 & 0.38 & 0.66 & 0.64 & 0.80 & 0.90 & 1.00 & 0.66 & 0.80 & 0.82 & 0.96 & \textbf{1.00} & 1.00 & \underline{0.94} & 1.00 \\
& 2.00 & 0.12 & 0.24 & 0.84 & 0.98 & 0.34 & 0.66 & 0.72 & 0.88 & \underline{0.96} & 1.00 & 0.68 & 0.86 & 0.82 & 0.98 & \textbf{1.00} & 1.00 & \underline{0.96} & 1.00 \\
\addlinespace[2pt]  % small vertical space after this group
\multirow{4}{*}{3} 
& 1.25 & 0.30 & 0.50 & 0.84 & 0.92 & 0.70 & 0.88 & 0.80 & 0.92 & \textbf{0.96} & 0.98 & 0.84 & 0.94 & 0.86 & 0.98 & \underline{0.94} & 1.00 & 0.88 & 0.96 \\
& 1.50 & 0.10 & 0.28 & 0.84 & 0.96 & 0.60 & 0.78 & 0.66 & 0.86 & \underline{0.90} & 1.00 & 0.72 & 0.86 & 0.84 & 0.98 & \textbf{0.96} & 1.00 & 0.86 & 0.96 \\
& 1.75 & 0.10 & 0.28 & \underline{0.88} & 1.00 & 0.42 & 0.78 & 0.72 & 0.88 & 0.76 & 0.94 & 0.74 & 0.88 & \underline{0.88} & 1.00 & \textbf{1.00} & 1.00 & \underline{0.88} & 0.98 \\
& 2.00 & 0.02 & 0.12 & \underline{0.86} & 0.94 & 0.38 & 0.52 & 0.60 & 0.78 & 0.70 & 0.94 & 0.60 & 0.80 & 0.82 & 0.92 & \textbf{0.98} & 1.00 & \underline{0.86} & 0.98 \\
\addlinespace[2pt]  % small vertical space after this group
\multirow{4}{*}{4} 
& 1.25 & 0.34 & 0.50 & 0.70 & 0.86 & 0.68 & 0.88 & 0.60 & 0.78 & 0.70 & 0.90 & 0.64 & 0.80 & 0.70 & 0.86 & \textbf{0.86} & 0.94 & \underline{0.72} & 0.88 \\
& 1.50 & 0.22 & 0.38 & 0.64 & 0.80 & 0.60 & 0.92 & 0.52 & 0.64 & 0.68 & 0.90 & 0.52 & 0.66 & 0.64 & 0.82 & \textbf{0.86} & 0.98 & \underline{0.72} & 0.82 \\
& 1.75 & 0.16 & 0.30 & 0.72 & 0.92 & 0.56 & 0.80 & 0.60 & 0.78 & 0.56 & 0.80 & 0.56 & 0.80 & 0.70 & 0.88 & \textbf{0.92} & 1.00 & \underline{0.80} & 0.94 \\
& 2.00 & 0.06 & 0.12 & 0.58 & 0.78 & 0.50 & 0.72 & 0.46 & 0.64 & 0.50 & 0.82 & 0.50 & 0.64 & 0.56 & 0.82 & \textbf{0.88} & 0.98 & \underline{0.64} & 0.88 \\
\bottomrule
\end{tabular}}
\label{tab:gunpoint_accuracy}
\end{table*}
\noindent\RuninHead{Results of GunPoint dataset} We utilize the GunPoint dataset to evaluate how the parameters $P$ and $l$ affect the ability of PSD to achieve invariance under piecewise scaling. 
The results are presented in Table~\ref{tab:gunpoint_accuracy}, where the best performance of $\overbar{P@1}$ is highlighted in bold, and the second-best is \underline{underlined}.
We benchmark our method against several state-of-the-art distance measures, including ADTW~\cite{herrmann2023amercing}, DDTW~\cite{keogh2001derivative}, shapeDTW~\cite{zhao2018shapedtw}, WDDTW~\cite{jeong2011weighted}, and WDTW~\cite{jeong2011weighted},. 
PSED achieves the highest accuracy, followed closely by PSDTW.
We attribute PSED's superior performance over PSDTW to the specific nature of the distortions in the query set, that is, the ``pure'' piecewise scaling distortions. 
Since the query set exhibits only piecewise scaling distortions, the corresponding segments of the query and target time series differ solely in length (scale). 
Consequently, the additional flexibility provided by DTW in PSDTW is unnecessary and may inadvertently increase the similarity of incorrect matches, thereby reducing discriminative power relative to the stricter PSED measure.
However, PSDTW remains theoretically essential for handling local distortions within the phase.
PSDTW applies DTW within the scaled segment, enabling robust alignment across local nonlinearities.

\begin{table*}[ht]
\centering
\caption{The accuracy comparison for six PSED-guided distance measures of the GunPoint dataset}
\setlength{\tabcolsep}{4pt} % (Default is 6pt.)
\begin{tabular}{cc|ccccccccccccccc}
\toprule
\multirow{2}{*}{$P$} & \multirow{2}{*}{$l$} 
& \multicolumn{2}{c}{DTW\textsuperscript{PSED}} 
& \multicolumn{2}{c}{ADTW\textsuperscript{PSED}} 
& \multicolumn{2}{c}{DDTW\textsuperscript{PSED}} 
& \multicolumn{2}{c}{shapeDTW\textsuperscript{PSED}} 
& \multicolumn{2}{c}{WDDTW\textsuperscript{PSED}} 
& \multicolumn{2}{c}{WDTW\textsuperscript{PSED}} \\
% \cmidrule(lr){3-4} \cmidrule(lr){5-6}
& & 
$\overline{P@1}$ & $\overline{P@3}$ &
$\overline{P@1}$ & $\overline{P@3}$ & 
$\overline{P@1}$ & $\overline{P@3}$ & 
$\overline{P@1}$ & $\overline{P@3}$ & 
$\overline{P@1}$ & $\overline{P@3}$ & 
$\overline{P@1}$ & $\overline{P@3}$ \\
\midrule
\multirow{4}{*}{2} 
& 1.25 & 0.86 & 1.00 & 0.98 & 1.00 & \textbf{0.72} & 0.92 & 0.98 & 1.00 & \textbf{0.74} & 0.96 & 0.86 & 1.00 \\
& 1.50 & 0.88 & 1.00 & 1.00 & 1.00 & \textbf{0.60} & 0.90 & \textbf{0.98} & 1.00 & \textbf{0.64} & 0.90 & 0.92 & 1.00 \\
& 1.75 & 0.94 & 1.00 & 1.00 & 1.00 & 0.84 & 0.94 & 1.00 & 1.00 & 0.86 & 0.96 & 0.94 & 1.00 \\
& 2.00 & 0.98 & 1.00 & 1.00 & 1.00 & \textbf{0.60} & 0.78 & 0.98 & 1.00 & \textbf{0.66} & 0.80 & 0.98 & 1.00 \\
\addlinespace[2pt]  % small vertical space after this group
\multirow{4}{*}{3} 
& 1.25 & \textbf{0.82} & 0.96 & 0.94 & 1.00 & \textbf{0.74} & 0.90 & \textbf{0.92} & 0.98 & \textbf{0.76} & 0.90 & \textbf{0.82} & 0.96 \\
& 1.50 & 0.86 & 1.00 & 0.96 & 1.00 & 0.72 & 0.90 & 0.94 & 1.00 & 0.80 & 0.90 & 0.88 & 1.00 \\
& 1.75 & 0.88 & 1.00 & 1.00 & 1.00 & \textbf{0.68} & 0.90 & 1.00 & 1.00 & \textbf{0.70} & 0.94 & 0.90 & 1.00 \\
& 2.00 & 0.86 & 0.96 & 0.98 & 1.00 & \textbf{0.54} & 0.86 & 0.96 & 1.00 & 0.62 & 0.86 & 0.88 & 0.96 \\
\addlinespace[2pt]  % small vertical space after this group
\multirow{4}{*}{4} 
& 1.25 & 0.74 & 0.88 & 0.86 & 0.94 & 0.78 & 0.80 & 0.80 & 0.92 & 0.78 & 0.82 & 0.74 & 0.88 \\
& 1.50 & 0.72 & 0.84 & 0.86 & 0.98 & 0.68 & 0.84 & 0.84 & 0.96 & 0.66 & 0.84 & 0.72 & 0.86 \\
& 1.75 & 0.74 & 0.96 & 0.92 & 1.00 & \textbf{0.54} & 0.90 & 0.90 & 1.00 & 0.56 & 0.88 & 0.74 & 0.96 \\
& 2.00 & 0.66 & 0.90 & 0.88 & 0.98 & 0.58 & 0.82 & 0.88 & 0.98 & 0.62 & 0.80 & 0.66 & 0.90 \\
\addlinespace[2pt]  % small vertical space after this group
\bottomrule
\end{tabular}
\label{tab:gunpoint_PSED-guided_accuracy}
\end{table*}
We further investigate whether the segmentation results returned by PSED can serve as a guide to enhance other distance measures. 
As shown in Table~\ref{tab:gunpoint_PSED-guided_accuracy}, this approach generally improves accuracy. 
The exceptions are highlighted in bold, indicating cases where the segmentation led to worse performance. 
%
Notably, in most cases, only DDTW and WDDTW failed to benefit from PSED-guided segmentation.
We argue that the performance degradation of DDTW and WDDTW stems from their derivative-based nature. 
They rely on matching slope or shape features.
Consequently, they are highly sensitive to segmentation boundaries. 
A non-perfect cut that falls within a shape feature segmentize it, and these features are then destroyed.
When the algorithm subsequently attempts to map these features, it fails to find correct correspondences, resulting in high distances.

% \begin{figure*}[t]
%     \centering
%     % Adjust width: 0.155\textwidth × 6 = about full page width
%     \begin{subfigure}{0.31\textwidth}
%         \centering
%         \includegraphics[width=\linewidth]{figures/runtime_P2_distcall.pdf}
%         \caption{}
%         \label{fig:ex1}
%     \end{subfigure}
%     \begin{subfigure}{0.31\textwidth}
%         \centering
%         \includegraphics[width=\linewidth]{figures/runtime_P3_distcall.pdf}
%         \caption{}
%         \label{fig:ex2}
%     \end{subfigure}
%     \begin{subfigure}{0.31\textwidth}
%         \centering
%         \includegraphics[width=\linewidth]{figures/runtime_P4_distcall.pdf}
%         \caption{}
%         \label{fig:ex3}
%     \end{subfigure}
%     \caption{Overall caption describing all 6 subfigures.}
%     \label{fig:six-wide}
% \end{figure*}

% \begin{figure*}[t]
%     \centering
%     % Adjust width: 0.155\textwidth × 6 = about full page width
%     \begin{subfigure}{0.31\textwidth}
%         \centering
%         \includegraphics[width=\linewidth]{figures/runtime_P2_psed_second.pdf}
%         \caption{}
%         \label{fig:ex1}
%     \end{subfigure}
%     \begin{subfigure}{0.31\textwidth}
%         \centering
%         \includegraphics[width=\linewidth]{figures/runtime_P3_psed_second.pdf}
%         \caption{}
%         \label{fig:ex2}
%     \end{subfigure}
%     \begin{subfigure}{0.31\textwidth}
%         \centering
%         \includegraphics[width=\linewidth]{figures/runtime_P3_psed_second.pdf}
%         \caption{}
%         \label{fig:ex3}
%     \end{subfigure}
%     \caption{Overall caption describing all 6 subfigures.}
%     \label{fig:six-wide}
% \end{figure*}

% \begin{figure*}[t]
%     \centering
%     % Adjust width: 0.155\textwidth × 6 = about full page width
%     \begin{subfigure}{0.31\textwidth}
%         \centering
%         \includegraphics[width=\linewidth]{figures/runtime_P2_second.pdf}
%         \caption{}
%         \label{fig:ex1}
%     \end{subfigure}
%     \begin{subfigure}{0.31\textwidth}
%         \centering
%         \includegraphics[width=\linewidth]{figures/runtime_P3_second.pdf}
%         \caption{}
%         \label{fig:ex2}
%     \end{subfigure}
%     \begin{subfigure}{0.31\textwidth}
%         \centering
%         \includegraphics[width=\linewidth]{figures/runtime_P4_second.pdf}
%         \caption{}
%         \label{fig:ex3}
%     \end{subfigure}
%     \caption{Overall caption describing all 6 subfigures.}
%     \label{fig:six-wide}
% \end{figure*}

% \begin{figure*}[htbp]
%     \centering
%     % --- Row 1 ---
%     % Row Caption (Left)
%     % \parbox[c]{0.03\linewidth}{\centering \rotatebox{90}{\textbf{Row 1 Caption}}}%
%     \parbox[c]{0.03\linewidth}{\centering \rotatebox{90}{$P = 1$}}%
%     \hfill
%     % Images
%     \subfloat[]{ \includegraphics[width=0.21\linewidth]{figures/runtime-P2-time.pdf} } \hfill
%     \subfloat[]{ \includegraphics[width=0.21\linewidth]{figures/runtime-P2-focus-time.pdf} } \hfill
%     \subfloat[]{ \includegraphics[width=0.21\linewidth]{figures/runtime-P2-distance-call.pdf} } \hfill
%     \subfloat[]{ \includegraphics[width=0.21\linewidth]{figures/runtime-P2-focus-distance-call.pdf} }
    
%     \vspace{0.5em} % Spacing between rows
%     \par \vspace{0.5cm}
    
%     % --- Row 2 ---
%     % Row Caption (Left)
%     \parbox[c]{0.03\linewidth}{\centering \rotatebox{90}{\textbf{Row 2 Caption}}}%
%     \hfill
%     % Images
%     \subfloat[]{ \includegraphics[width=0.21\linewidth]{figures/runtime-P3-time.pdf} } \hfill
%     \subfloat[R2-C2]{ \includegraphics[width=0.21\linewidth]{figures/runtime-P3-focus-time.pdf} } \hfill
%     \subfloat[R2-C3]{ \includegraphics[width=0.21\linewidth]{figures/runtime-P3-distance-call.pdf} } \hfill
%     \subfloat[R2-C4]{ \includegraphics[width=0.21\linewidth]{figures/runtime-P3-focus-distance-call.pdf} }
    
%     \vspace{0.5em} % Spacing between rows
%     \par \vspace{0.5cm}
    
%     % --- Row 3 ---
%     % Row Caption (Left)
%     \parbox[c]{0.03\linewidth}{\centering \rotatebox{90}{\textbf{Row 3 Caption}}}%
%     \hfill
%     % Images
%     \subfloat[R3-C1]{ \includegraphics[width=0.21\linewidth]{figures/runtime-P4-time.pdf} } \hfill
%     \subfloat[R3-C2]{ \includegraphics[width=0.21\linewidth]{figures/runtime-P4-focus-time.pdf} } \hfill
%     \subfloat[R3-C3]{ \includegraphics[width=0.21\linewidth]{figures/runtime-P4-distance-call.pdf} } \hfill
%     \subfloat[R3-C4]{ \includegraphics[width=0.21\linewidth]{figures/runtime-P4-focus-distance-call.pdf} }

%     % Overall Figure Caption
%     \caption{Overall caption describing the entire 3x4 grid comparison.}
%     \label{fig:grid_comparison}
% \end{figure*}

% \begin{figure*}[htbp]
%     \centering
    
%     % --- Row 1 (P=1) ---
%     % Row Label (Rotated Left)
%     \parbox[c]{0.03\linewidth}{\centering \rotatebox{90}{$P = 1$}}%
%     \hfill
%     % Images
%     \subfloat[]{ \includegraphics[width=0.21\linewidth]{figures/runtime-P2-time.pdf} } \hfill
%     \subfloat[]{ \includegraphics[width=0.21\linewidth]{figures/runtime-P2-focus-time.pdf} } \hfill
%     \subfloat[]{ \includegraphics[width=0.21\linewidth]{figures/runtime-P2-distance-call.pdf} } \hfill
%     \subfloat[]{ \includegraphics[width=0.21\linewidth]{figures/runtime-P2-focus-distance-call.pdf} }
    
%     \par \vspace{0.2cm} % Single consistent vertical space
    
%     % --- Row 2 (P=2) ---
%     \parbox[c]{0.03\linewidth}{\centering \rotatebox{90}{$P = 2$}}%
%     \hfill
%     % Images
%     \subfloat[]{ \includegraphics[width=0.21\linewidth]{figures/runtime-P3-time.pdf} } \hfill
%     \subfloat[]{ \includegraphics[width=0.21\linewidth]{figures/runtime-P3-focus-time.pdf} } \hfill
%     \subfloat[]{ \includegraphics[width=0.21\linewidth]{figures/runtime-P3-distance-call.pdf} } \hfill
%     \subfloat[]{ \includegraphics[width=0.21\linewidth]{figures/runtime-P3-focus-distance-call.pdf} }
    
%     \par \vspace{0.2cm} % Single consistent vertical space
    
%     % --- Row 3 (P=3) ---
%     \parbox[c]{0.03\linewidth}{\centering \rotatebox{90}{$P = 3$}}%
%     \hfill
%     % Images
%     \subfloat[]{ \includegraphics[width=0.21\linewidth]{figures/runtime-P4-time.pdf} } \hfill
%     \subfloat[]{ \includegraphics[width=0.21\linewidth]{figures/runtime-P4-focus-time.pdf} } \hfill
%     \subfloat[]{ \includegraphics[width=0.21\linewidth]{figures/runtime-P4-distance-call.pdf} } \hfill
%     \subfloat[]{ \includegraphics[width=0.21\linewidth]{figures/runtime-P4-focus-distance-call.pdf} }

%     % Overall Figure Caption
%     \caption{Runtime comparison across different P values. (a)-(d) $P=1$, (e)-(h) $P=2$, (i)-(l) $P=3$.}
%     \label{fig:grid_comparison}
% \end{figure*}

\begin{figure*}[htbp]
    \centering
    
    % --- Row 1 (P=1) ---
    % Label
    \parbox[c]{0.03\linewidth}{\centering \rotatebox{90}{$P = 1$}}%
    \hfill
    % Images with valign=c
    \subfloat[]{ \includegraphics[width=0.21\linewidth, valign=c]{figures/gunpoint-runtime-P2-time.pdf} } \hfill
    \subfloat[]{ \includegraphics[width=0.21\linewidth, valign=c]{figures/gunpoint-runtime-P2-focus-time.pdf} } \hfill
    \subfloat[]{ \includegraphics[width=0.21\linewidth, valign=c]{figures/gunpoint-runtime-P2-distance-call.pdf} } \hfill
    \subfloat[]{ \includegraphics[width=0.21\linewidth, valign=c]{figures/gunpoint-runtime-P2-focus-distance-call.pdf} }
    
    \par \vspace{0.2cm} 
    
    % --- Row 2 (P=2) ---
    % Label
    \parbox[c]{0.03\linewidth}{\centering \rotatebox{90}{$P = 2$}}%
    \hfill
    % Images with valign=c
    \subfloat[]{ \includegraphics[width=0.21\linewidth, valign=c]{figures/gunpoint-runtime-P3-time.pdf} } \hfill
    \subfloat[]{ \includegraphics[width=0.21\linewidth, valign=c]{figures/gunpoint-runtime-P3-focus-time.pdf} } \hfill
    \subfloat[]{ \includegraphics[width=0.21\linewidth, valign=c]{figures/gunpoint-runtime-P3-distance-call.pdf} } \hfill
    \subfloat[]{ \includegraphics[width=0.21\linewidth, valign=c]{figures/gunpoint-runtime-P3-focus-distance-call.pdf} }
    
    \par \vspace{0.2cm} 
    
    % --- Row 3 (P=3) ---
    % Label
    \parbox[c]{0.03\linewidth}{\centering \rotatebox{90}{$P = 3$}}%
    \hfill
    % Images with valign=c
    \subfloat[]{ \includegraphics[width=0.21\linewidth, valign=c]{figures/gunpoint-runtime-P4-time.pdf} } \hfill
    \subfloat[]{ \includegraphics[width=0.21\linewidth, valign=c]{figures/gunpoint-runtime-P4-focus-time.pdf} } \hfill
    \subfloat[]{ \includegraphics[width=0.21\linewidth, valign=c]{figures/gunpoint-runtime-P4-distance-call.pdf} } \hfill
    \subfloat[]{ \includegraphics[width=0.21\linewidth, valign=c]{figures/gunpoint-runtime-P4-focus-distance-call.pdf} }

    \caption{Runtime and number of distance calls comparison of GunPoint dataset. (a)-(d) $P=1$, (e)-(h) $P=2$, (i)-(l) $P=3$.}
    \label{fig:gunpoint-runtime}
\end{figure*}
Figure~\ref{fig:gunpoint-runtime} illustrates the runtime performance across varying parameters $P$ and $l$. 
We evaluate two variants of PSD, PSED and PSDTW, each implemented with three levels of 
\begin{enumerate}
    \item \texttt{vanilla} (i.e., Basic implementation)
    \item \texttt{parallel\_bsf} (i.e., With parallelization with Best-So-Far early abandoning)
    \item \texttt{parallel\_bsf\_lb} (i.e., Incorporating lower bound pruning)
\end{enumerate}

This yields a total of six methods.
The figure is organized into four columns plotted against the scaling factor $l$. 
The rows refer to the number of pieces $P$.
The first and third columns display the running time and the number of distance calculations, respectively, for all six methods. 
To better visualize the performance differences among the efficient implementations, the second and fourth columns omit the \texttt{vanilla} baselines and focus exclusively on the four optimized variants.

The results indicate that \texttt{vanilla\_PSDTW} is orders of magnitude slower than the other approaches, whereas the optimized methods operate within a similar performance range. 
As anticipated, the computation time for all methods increases with the scaling factor $l$.
The fourth column confirms that the lower bounding strategy effectively reduces the number of distance calculations (pruning power). 
However, the second column reveals a critical trade-off in actual runtime.
While the lower bound successfully accelerates PSDTW, it actually slows down PSED. 
This suggests that for the computationally lighter ED, the overhead of calculating the lower bound outweighs the time saved by pruning. 
Overall, the results demonstrate that PSED is significantly faster than PSDTW.

\noindent\RuninHead{Results of the ten datasets}
\begin{table*}[ht]
\centering
\caption{The accuracy comparison for eight distance measures of the ten datasets}
\setlength{\tabcolsep}{4pt} % (Default is 6pt.)
\begin{tabular}{c|cc|cccccccccccc|cc}
\toprule
\multirow{2}{*}{Dataset}
& \multicolumn{2}{c|}{ED}
& \multicolumn{2}{c}{DTW} 
& \multicolumn{2}{c}{ADTW} 
& \multicolumn{2}{c}{DDTW} 
& \multicolumn{2}{c}{shapeDTW} 
& \multicolumn{2}{c}{WDDTW} 
& \multicolumn{2}{c}{WDTW} 
& \multicolumn{2}{|c}{PSED} \\
% \cmidrule(lr){3-4} \cmidrule(lr){5-6}
& 
$\overline{P@1}$ & $\overline{P@3}$ &
$\overline{P@1}$ & $\overline{P@3}$ & 
$\overline{P@1}$ & $\overline{P@3}$ & 
$\overline{P@1}$ & $\overline{P@3}$ & 
$\overline{P@1}$ & $\overline{P@3}$ & 
$\overline{P@1}$ & $\overline{P@3}$ & 
$\overline{P@1}$ & $\overline{P@3}$ & 
$\overline{P@1}$ & $\overline{P@3}$ \\
\midrule
SonyAIBORobotSurface1 & 0.04 & 0.08 & 0.51 & 0.63 & \underline{0.60} & 0.74 & 0.45 & 0.57 & 0.20 & 0.29 & 0.45 & 0.58 & 0.51 & 0.64 & \textbf{0.67} & 0.77 \\
ECG200 & 0.12 & 0.20 & 0.70 & 0.81 & \underline{0.78} & 0.83 & 0.60 & 0.74 & 0.59 & 0.76 & 0.59 & 0.73 & 0.70 & 0.80 & \textbf{0.81} & 0.88 \\
MedicalImages & 0.08 & 0.19 & 0.63 & 0.78 & 0.69 & 0.85 & 0.52 & 0.68 & 0.57 & 0.75 & 0.51 & 0.68 & 0.62 & 0.78 & \textbf{0.76} & 0.88 \\
CBF & 0.53 & 0.67 & 0.83 & 1.00 & \underline{0.90} & 1.00 & 0.73 & 0.87 & 0.73 & 0.90 & 0.77 & 0.90 & \underline{0.90} & 1.00 & \textbf{1.00} & 1.00 \\
SwedishLeaf & 0.07 & 0.12 & 0.82 & 0.92 & \underline{0.84} & 0.93 & 0.70 & 0.83 & 0.78 & 0.91 & 0.69 & 0.83 & 0.83 & 0.92 & \textbf{0.97} & 0.99 \\
Plane & 0.09 & 0.14 & 0.50 & 0.74 & \underline{0.59} & 0.78 & 0.38 & 0.64 & 0.53 & 0.79 & 0.37 & 0.61 & 0.49 & 0.75 & \textbf{0.75} & 0.93 \\
PowerCons & 0.26 & 0.43 & 0.77 & 0.98 & \textbf{0.80} & 1.00 & 0.77 & 0.98 & 0.77 & 0.99 & 0.77 & 0.99 & 0.77 & 0.99 & \textbf{0.80} & 1.00 \\
GunPoint & 0.10 & 0.28 & \underline{0.84} & 0.96 & 0.60 & 0.78 & 0.66 & 0.86 & 0.90 & 1.00 & 0.72 & 0.86 & \underline{0.84} & 0.98 & \textbf{0.96} & 1.00 \\
Adiac & 0.01 & 0.02 & \underline{0.37} & 0.50 & 0.12 & 0.21 & 0.32 & 0.42 & 0.27 & 0.41 & 0.31 & 0.41 & 0.36 & 0.50 & \textbf{0.57} & 0.71 \\
Epilepsy & 0.18 & 0.26 & 0.74 & 0.88 & \underline{0.80} & 0.91 & 0.65 & 0.82 & 0.38 & 0.47 & 0.58 & 0.79 & 0.70 & 0.83 & \textbf{0.83} & 0.91 \\

\bottomrule
\end{tabular}
\label{tab:ten_datasets_accuracy}
\end{table*}
\begin{table*}[ht]
\centering
\caption{The accuracy comparison for six PSED-guided distance measures of the ten datasets}
\setlength{\tabcolsep}{4pt} % (Default is 6pt.)
\resizebox{1.0\columnwidth}{!}{
\begin{tabular}{c|cccccccccccc}
\toprule
\multirow{2}{*}{Dataset}
& \multicolumn{2}{c}{DTW\textsuperscript{PSED}} 
& \multicolumn{2}{c}{ADTW\textsuperscript{PSED}} 
& \multicolumn{2}{c}{DDTW\textsuperscript{PSED}} 
& \multicolumn{2}{c}{shapeDTW\textsuperscript{PSED}} 
& \multicolumn{2}{c}{WDDTW\textsuperscript{PSED}} 
& \multicolumn{2}{c}{WDTW\textsuperscript{PSED}} \\
% \cmidrule(lr){3-4} \cmidrule(lr){5-6}
& 
$\overline{P@1}$ & $\overline{P@3}$ &
$\overline{P@1}$ & $\overline{P@3}$ & 
$\overline{P@1}$ & $\overline{P@3}$ & 
$\overline{P@1}$ & $\overline{P@3}$ & 
$\overline{P@1}$ & $\overline{P@3}$ & 
$\overline{P@1}$ & $\overline{P@3}$ \\
\midrule
SonyAIBORobotSurface1 & 0.53 & 0.64 & 0.64 & 0.76 & \textbf{0.37} & 0.53 & 0.65 & 0.76 & \textbf{0.38} & 0.53 & 0.53 & 0.65 \\
ECG200 & 0.73 & 0.84 & 0.80 & 0.88 & \textbf{0.54} & 0.68 & 0.81 & 0.88 & \textbf{0.53} & 0.69 & 0.73 & 0.84 \\
MedicalImages & 0.66 & 0.78 & 0.76 & 0.88 & 0.51 & 0.69 & 0.74 & 0.88 & 0.51 & 0.70 & 0.66 & 0.79 \\ 
CBF & 0.90 & 1.00 & 0.93 & 1.00 & 0.73 & 0.93 & 0.93 & 0.97 & \textbf{0.70} & 0.93 & 0.90 & 1.00 \\
SwedishLeaf & 0.82 & 0.92 & 0.97 & 0.99 & 0.72 & 0.87 & 0.97 & 0.99 & 0.73 & 0.88 & \textbf{0.82} & 0.92 \\ 
Plane & 0.60 & 0.80 & 0.75 & 0.94 & 0.50 & 0.67 & 0.74 & 0.90 & 0.50 & 0.69 & 0.59 & 0.80 \\
GunPoint & 0.86 & 1.00 & 0.96 & 1.00 & 0.72 & 0.90 & 0.94 & 1.00 & 0.80 & 0.90 & 0.88 & 1.00 \\
PowerCons & 0.78 & 0.99 & 0.79 & 1.00 & 0.79 & 1.00 & 0.79 & 1.00 & 0.79 & 1.00 & 0.78 & 1.00 \\
Adiac & \textbf{0.33} & 0.45 & 0.57 & 0.71 & \textbf{0.21} & 0.34 & 0.53 & 0.70 & \textbf{0.20} & 0.33 & \textbf{0.34} & 0.45 \\
Epilepsy & 0.77 & 0.88 & \textbf{0.78} & 0.88 & 0.66 & 0.79 & 0.74 & 0.89 & 0.66 & 0.82 & 0.80 & 0.89 \\ 

\bottomrule
\end{tabular}}
\label{tab:ten_datasets_PSED-guided_accuracy}
\end{table*}
% \begin{table*}[ht]
% \centering
% \caption{The number of distance calls and runtime on PSED\_vanilla and PSED\_parallel\_bsf of the ten datasets}
% % Note: The column specification now has 7 columns: l (Name) + c c (Size, Length) + (c c) * 2 (PSED groups)
% \begin{tabular}{l c c c c c c}
% \toprule
% \multirow{2}{*}{Name} & \multirow{2}{*}{Size} & \multirow{2}{*}{Length} & \multicolumn{2}{c}{PSED\_vanilla} & \multicolumn{2}{c}{PSED\_parallel\_bsf} \\
% \cmidrule(lr){4-5} \cmidrule(lr){6-7}
% & & & \# distance calls & Time (s) & \# distance calls & Time (s)\\
% \midrule
% % Original PSED Data
% SonyAIBORobotSurface1 & 601 & 70 & 3212093055 & 697 & 308323603 & 69 \\
% ECG200 & 100 & 96 & 367078356 & 91 & 32746898 & 3 \\
% MedicalImages & 381 & 99 & 8449375162 & 2082 & 335106685 & 40 \\
% CBF & 30 & 128 & 144715449 & 43 & 35732630 & 2 \\
% SwedishLeaf & 500 & 128 & 34180325786 & 16601 & 1286324350 & 107 \\
% Plane & 105 & 144 & 2020267770 & 621 & 79732626 & 6 \\
% PowerCons & 180 & 144 & 6698798003 & 3629 & 1184577517 & 49 \\
% GunPoint & 50 & 150 & 659109745 & 159 & 69752469 & 3 \\
% Adiac & 390 & 175 & 54549344741 & 23550 & 1793395497 & 123 \\
% Epilepsy & 137 & 206 & 18885650765 & 13131 & 11354378205 & 336 \\
% \bottomrule
% \end{tabular}
% \label{tab:ten_datasets_efficiency_two_sets}
% \end{table*}

\begin{table*}[ht]
\centering
\caption{The number of distance calls and runtime on PSED\_vanilla and PSED\_parallel\_bsf of the ten datasets}
% Note: The column specification has 7 columns
\begin{tabular}{l c c c c c c}
\toprule
\multirow{2}{*}{Name} & \multirow{2}{*}{Size} & \multirow{2}{*}{Length} & PSED\_vanilla & PSED\_parallel\_bsf & \multirow{2}{*}{\% distance calls pruned} & \multirow{2}{*}{Speed Up} \\
\cmidrule(lr){4-4} \cmidrule(lr){5-5}
& & & Time (s) & Time (s) & & \\
\midrule
% Original PSED Data
SonyAIBORobotSurface1 & 601 & 70 & 697 & 69 & 90.40\% & 10.10X \\
ECG200 & 100 & 96 & 91 & 3 & 91.08\% & 30.33X \\
MedicalImages & 381 & 99 & 2082 & 40 & 96.03\% & 52.05X \\
CBF & 30 & 128 & 43 & 2 & 75.31\% & 21.50X \\
SwedishLeaf & 500 & 128 & 16601 & 107 & 96.24\% & 155.15X \\
Plane & 105 & 144 & 621 & 6 & 96.05\% & 103.50X \\
PowerCons & 180 & 144 & 3629 & 49 & 82.32\% & 74.06X \\
GunPoint & 50 & 150 & 159 & 3 & 89.42\% & 53.00X \\
Adiac & 390 & 175 & 23550 & 123 & 96.71\% & 191.46X \\
Epilepsy & 137 & 206 & 13131 & 336 & 39.88\% & 39.08X \\
\bottomrule
\end{tabular}
\label{tab:ten_datasets_efficiency}
\end{table*}

For the remaining nine datasets, we fix the parameters at $P=3$ and $l=1.5$. 
We have the following findings from the previous experiment:
\begin{enumerate}
    \item From Table~\ref{tab:gunpoint_PSED-guided_accuracy}, PSED outperformed PSDTW in handling piecewise scaling distortions.
    \item From Figure~\ref{fig:gunpoint-runtime}, the lower bound offered no efficiency gain for PSED.
\end{enumerate}

Hence, we select \texttt{PSED\_parallel\_bsf} as the representative method for this evaluation.
The accuracy results are presented in Table~\ref{tab:ten_datasets_accuracy}, while the results for the PSED-guided distance measures are detailed in Table~\ref{tab:ten_datasets_PSED-guided_accuracy}. 
Finally, the runtime efficiency for all datasets is summarized in Table~\ref{tab:ten_datasets_efficiency}.
It shows a significant speedup, ranging from 10.10X to 191.46X.

