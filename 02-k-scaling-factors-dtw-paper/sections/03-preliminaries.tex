\section{Preliminaries} \label{sec:preliminaries}

We refer to time as the contextual attribute because it provides the context for the measurements to be made.
We refer to the measurements as the behavioral attributes. 
Time series are multivariate when more than one behavioral attribute is present. Otherwise, it is called univariate.
We focus on the univariate case.

\begin{definition}[Time Series] 
    \label{def:ts}
    A time series $T = t_1, t_2, ..., t_n$ is a sequence of real-valued numbers with length = $n$.
\end{definition}

When two time series are involved in the discussion, we denote them as $Q$ (Query) and $C$ (Candidate), with lengths $m$ and $n$, respectively.
Since $Q$ is the query sequence, it is not longer than $C$ (i.e., $m \leq n$).
The requirement of ``$m \leq n$'' is a natural setting.
In ``Query by Content'', a user is going to search for a candidate in the database from a user-input query in which the query may only contain partial information of the target candidate.
For example, a user often wants to find a song or tune that is lingering in their head by humming a part but not whole of the tune~\cite{zhu2003warping}.
We are also interested in a segment or subsequence of a time series. 

\begin{definition}[Subsequence] 
    \label{def:subseq}
    A subsequence $T(i:j)$ of a time series $T$ is a shorter time series that starts from position $i$ and ends at position $j$ with length = $j - i + 1$. Both ends are inclusive.
    Formally, $T(i:j) = t_i, t_{i+1}, ..., t_j$, $1 \leq i \leq j \leq n$.
\end{definition}
We call $T(1:j)$ the prefix of $T$ of length $j$.

% Normalization
\begin{figure}[tb]
    \centering
    \includegraphics[width=1.0\columnwidth]{figures/normalization.pdf}
    \caption{Z-normalization. Resulting time series have mean $= 0$ and std $= 1$.}
    \label{fig:normalization}
\end{figure}
Before comparison, we need to standardize or normalize them.
A common way is Z-Normalization, which is $T = (T-\operatorname{mean}(T))/\operatorname{std}(T)$, as shown in Figure~\ref{fig:normalization}. 
The most fundamental distance measure is the Euclidean Distance (ED).

\subsection{Euclidean Distance (ED)}

Given two points on a plane, it is intuitive to define the distance between them as the length of the line segment between them. 
This idea extends to the case of $n$-dimensions in time series with length $n$.

\begin{definition}[Euclidean Distance (ED)]
    \label{def:ed}
    Given two series $Q$ and $C$ both with length $n$, the Euclidean Distance between them is defined as:
    \begin{equation} 
        \label{eq:ed}
        \operatorname{ED}(Q, C) = \sqrt{\sum_{i=1}^n (q_i-c_i)^2}
    \end{equation}
\end{definition}

A square root is usually involved in the computation of distance measures.
It is a monotonic function. Since it does not change the relative ranking of the results, we can omit the square root operation for simplicity and optimization.
ED aligns the entries between two series in a one-to-one manner.
Two similar series will have a large distance under ED if they are not aligned well in the time dimension.
ED cannot handle local distortion along the time axis because warping in the alignment is not allowed.
A standard distance measure that provides warping invariance is called Dynamic Time Warping (DTW).

\subsection{Dynamic Time Warping (DTW)}

Dynamic Time Warping (DTW) is an algorithm that measures similarity between two time series while accounting for local distortions.
DTW aligns the two series by nonlinearly warping the time axis to minimize the final cumulative distance.

Given two time series, $Q$ and $C$, we first construct an $m$ by $n$ distance matrix $M$.
The origin ($1$, $1$) is set at the bottom-left element of $M$.
The ($i$, $j$) element of $M$ contains the distance $\operatorname{d}(q_i, c_j)$ between points $q_i$ and $c_j$.
This local distance $\operatorname{d}(\cdot, \cdot)$ is usually calculated by $(q_i - c_j)^2$.
Each ($i$, $j$) element refers to an alignment or mapping of the two points.
A contiguous set of such elements forms a warping path $W$.
$W$ represents the non-linear alignment of $Q$ and $C$.
$W = w_1, w_2, ..., w_K$ in which $w_k = (i, j)_k$ represents the mapping between $q_i$ in $Q$ and $c_j$ in $C$, where $\max(m, n) \leq K \leq m + n -1$.
``$\max(m, n) \leq K$'' because the alignment of two series must include every point in both $Q$ and $C$.
``$K \leq m + n -1$'' because the longest warping path is either the ``concatenation of the bottom row and the rightmost column'' (with the bottom-right cell being overlapped) or the ``concatenation of the leftmost column and the top row'' (with the top-left cell being overlapped).

The warping path $W$ is typically subject to the following three constraints.
\begin{itemize}
  \item Boundary conditions: $w_1 = (1, 1)$ and $w_K = (m, n)$. The first (last) point of $Q$ must map to that of $C$.
  \item Continuity: Given $w_k = (i, j)$ and $w_{k-1} = (i', j')$, $i - i' \leq 1$ and $j - j' \leq 1$. An entry in the warping path $W$ is adjacent to its one-step previous entry.  
  % It restricts the allowable steps in the warping path to adjacent cells.
  \item Monotonicity: Given $w_k = (i, j)$ and $w_{k-1} = (i', j')$, $i - i' \geq 0$ and $j - j' \geq 0$. The warping path $W$ does not go back. 
  % It either goes up, down, or remains stationary with respect to the vertical axis.
\end{itemize}
We denote $W^*$ as a set of all allowed possible paths.

\begin{definition}[Dynamic Time Warping (DTW)~\cite{sakoe1978dynamic}]
    \label{def:dtw}
    DTW returns the minimum warping cost:
    \begin{equation} 
        \label{eq:dtw}
        \operatorname{DTW}(Q, C) = \min_{W \in W^*}{\sum_{k=1}^{|W|} w_k}
    \end{equation}
\end{definition}
To find the minimum warping cost and its corresponding warping path, we can use dynamic programming (DP) to evaluate the following recurrence.

\begin{equation}
\label{eq:dtw-recurrence}
\begin{split}
    D(i, j) &= \operatorname{d}(q_i, c_j) \\
    &\quad + \min \begin{cases}
        D(i-1, j-1), \\
        D(i-1, j), \\
        D(i, j-1)
    \end{cases}
\end{split}
\end{equation}

It can be solved by building the accumulated cost matrix $D$, where the y-axis refers to $Q$, and the x-axis refers to $C$.
The base cases, which are the first row and the first column, are defined as $D(1, j) = \sum_{k=1}^{j} \operatorname{d}(q_1, c_k), j \in [1, n]$ and $D(i, 1) = \sum_{k=1}^{i} d(q_k, c_1), i \in [1, m]$. 
After we have initialized the base cases, we can fill up $D$ starting from the bottom up.
There are $m \times n$ entries in $D$. It takes $\mathcal{O}(mn)$ to fill it up.
Once $D$ is built, we can find the path corresponding to this minimum cost by simple backtracking from the end cell $D(m, n)$ to the origin cell $D(1, 1)$.

\begin{figure}[tb]
    \centering
    \includegraphics[width=1.0\columnwidth]{figures/dtw-matrix.pdf}
    \caption{Visualization of $D$ with local constraints. The black cells form the warping path.}
    \label{fig:dtw-matrix}
\end{figure}
Some constraints have been proposed to prevent pathological warping paths with the aim of accuracy and efficiency.
Two of the most popular are Sakoe-Chiba Band~\cite{sakoe1978dynamic} and Itakura Parallelogram~\cite{itakura1975minimum}.
They only allow the warping paths to pass through the allowed region, as shown in Figure~\ref{fig:dtw-matrix}, by setting the cells outside this region in $D$ to have $\infty$ accumulated distance cost.
\cite{keogh2005exact} suggested that these constraints can be viewed as constraints on the warping path entry $w_k = (i,j)_k$.
It represents these constraints as inequalities applying to the indices $i$ and $j$ locally, without depending on the main diagonal in $D$.
In the Sakoe-Chiba Band, the constraints can be represented as $j -r \leq i \leq j+r$, where $r$ is an integer.
It is sometimes specified as a fraction (or percentage) of the longer time series length to ensure length invariance.
For clarity, we assume $r$ is an integer unless specified otherwise.
This means that $q_i$ can only align with $c_j$ if their indices differ by at most $r$.
Since $r$ defines the maximum allowed difference between the mapping indices, $|n - m| \leq r$, to ensure that the end points of $Q$ and $C$ can map.
In the Itakura Parallelogram, $r$ is a function of $i$ rather than a constant value.
ED can be seen as a special case of DTW where the warping path is fixed to be diagonal. $q_i$ aligns to $c_i$ for every $i$ (i.e., $r = 0$).
%
DTW minimizes over all possible warping paths, and the warping path of ED is one of them.
Because of the band, we only need to fill up the cells within the band. The complexity is $\mathcal{O}(\#\_of\_cells\_inside)$.
In the case of the Sakoe-Chiba Band, the band has a constant width $w = 2r + 1$.
The complexity becomes $\mathcal{O}(w \times length\_of\_diagonal) = \mathcal{O}(rn)$.
The constrained DTW is denoted as $\operatorname{DTW}_r$.

\subsection{Uniform Scaling (US)}

% Given a query series $Q$ and a candidate series $C$, of length $m$ and $n$ respectively, where $m \leq n$.
In US, we compare the whole sequence of $Q$ to a prefix of $C$, as shown in Figure~\ref{fig:us-prefix}.
The two compared sequences are scaled to the same length via interpolation before ED is applied.
A common interpolation method is nearest neighbor interpolation.

\begin{definition}[Nearest Neighbor Interpolation]
    \label{def:interpolation}
    Given a time series $T$ of length $n$ and an integer $L$  , Nearest Neighbor Interpolation scales $T$ into $T^L$ as follows:
    \begin{equation} 
        \label{eq:interpolation}
        T^L_j = T_{\ceil{n(j/L)})} \quad where\; 1 \leq j \leq L
    \end{equation}
\end{definition}
We can scale up or down a given series using Equation~\ref{eq:interpolation}, as shown in Example~\ref{ex:interpolation}.

\begin{example}
    \label{ex:interpolation}
    Given a series $T = 1, 2, \cdots, 6$ with length $n = 6$. 
    Let $T(1:4)$ be the prefix of $T$ of length $k = 4$ (i.e., $T(1:4) = 1, 2, 3, 4$).
    Given an integer $L = 8$, we compute $T(1:4)^8$ as follows.
    \begin{align*}
        T(1:4)^8 &= T_{\ceil{4(1/8)}}, T_{\ceil{4(2/8)}}, T_{\ceil{4(3/8)}}, \dots, T_{\ceil{4(8/8)}} \\
                 &= T_1, T_1, T_2, \dots, T_4 \\
                 &= 1, 1, 2, \dots, 4.
    \end{align*}
\end{example}

When $L > k$, $T$ is said to be stretched. When $L < k$, $T$ is said to be shrunk.

\begin{definition}[Uniform Scaling (US)~\cite{keogh2004indexing}]
    \label{def:us}
    Given two series $Q$ and $C$, of length $m$ and $n$ respectively, and a scaling factor bound $l$, where $l \geq 1$.
    Let $C(1:k)$ be the prefix of $C$, where $\ceil{m/l} \leq k \leq \min(\floor{lm}, n)$, and ${C(1:k)}^L$ be a rescaled version of $C(1:k)$ with length $L$, where $L = \min (\floor{lm}, n)$. 
    $L$ is called the alignment factor.
    $\min (\floor{lm}, n)$ is the largest alignment factor.
    \begin{equation} 
        \label{eq:us}
        \operatorname{US}(Q,C,l,L) = \min_{k=\ceil{m/l}}^{\min (\floor{lm}, n)} \operatorname{ED}(Q^L, {C(1:k)}^L)
    \end{equation}
\end{definition}
$L$ is set as the largest alignment factor~\cite{shen2018accelerating} to ensure all the points in $Q$ and $C(1:k)$ are preserved during interpolation because of up-sampling, and the scaled version of all the prefixes is going to have the same length (i.e., $L$), for fair comparison, since comparison between two longer time series generally results in a larger distance measure.
Through Equation~\ref{eq:us}, we find the minimum value and the corresponding argument (i.e., $k$) by checking the minimum value of the ED function from $\ceil{m/l}$ to $\min (\floor{lm}, n)$.  
The scaling factor is defined by the argument minimum value of $k$.
The smaller $k$ is, the more we need to ``stretch'' $C$ for $Q$ to compare with $C(1:k)$, which is $m/k$ times.

Consider a time series database $D$ comprising a set of candidate instances, and a single query series $Q$. The search task aims to retrieve the most similar instance (or top-$k$ instances) from $D$ to $Q$. 
We maintain $Q$ as a fixed reference and extract only the prefixes from each instance in $D$ for comparison. 
Furthermore, to simplify notation, we apply scaling exclusively to the instances in $D$, leaving $Q$ unscaled.
% However, the same methodology also works in an alternative way.

\subsection{Uniform Scaling \& Dynamic Time Warping (USDTW)}

Uniform Scaling \& Dynamic Time Warping (USDTW) measures the similarity by applying scaling with an appropriate scaling factor on $C$ and then applying DTW. 
The tail part of $C$ can be ignored, as shown in Figure~\ref{fig:us-prefix}.

\begin{definition}[Uniform Scaling \& Dynamic Time Warping (USDTW) \cite{shen2018accelerating}]
    \label{def:usdtw}
    With the same notations defined in Definition~\ref{def:us},
    \begin{equation} \label{eq:usdtw}
    \begin{split}
        \operatorname{USDTW}_r(Q,C,l, L) &= \\
        \min_{k=\ceil{m/l}}^{\min (\floor{lm}, n)} & \operatorname{DTW}_r(Q^L, {C(1:k)}^L)
    \end{split}
    \end{equation}
where $r$ is the DTW constraint parameter.
\end{definition}
We replace the ED function in Equation~\ref{eq:us} by DTW function to form Equation~\ref{eq:usdtw}.

As mentioned in Section~\ref{sec:introduction}, there may exist more than one scaling factor. 
Hence, both the existing distance measures, US and USDTW, which are designed to handle only one scaling factor, are insufficient.
This motivates us to design a new framework of distance measures.

\subsection{Lower Bounds for DTW and USDTW}

We first introduce the concept of lower bounds and explain how they can benefit search.
%
DTW is computationally more expensive than ED.
They compute an accumulated cost matrix of size $m \times n$, where $m$ and $n$ denote the lengths of the two time series. 
It results in quadratic complexity.
This complexity poses a challenge for similarity search.
The most common approach is to compute a lower bound of the real value, which is computationally cheap and tight.
We can use this lower bound to filter out unpromising candidates and perform the expensive distance computation only on the small set of promising candidates.
In searching, we would keep track of the best\_so\_far distance $\mathrm{bsf}$ between the query $Q$ and the testing candidates.
When testing a new candidate $C$, we first compute the $\operatorname{LB}(Q, C)$. We only compute the actual DTW when $\operatorname{LB}(Q, C) \leq \mathrm{bsf}$.

We then introduce some common lower bounds.

\noindent\RuninHead{Kim Lower Bound~\cite{kim2001indexbased}} $\lbKim$ is a simple and fast lower bound of DTW.
The complexity is $O(1)$.
It uses the four features in $Q$ and $C$.
We denote $t_{-1}$ as the last point and $t_{\max}$ ($t_{\min}$) as the maximum (minimum) point in time series $T$.

\begin{equation} 
    \label{eq:lb-kim}
    \lbKim= \max
        \begin{cases}
            \operatorname{d}(q_1, c_1)\\
            \operatorname{d}(q_{-1}, c_{-1})\\
            \operatorname{d}(q_{\max}, c_{\max})\\
            \operatorname{d}(q_{\min}, c_{\min)}\\
        \end{cases}
\end{equation}

$\operatorname{d}(\cdot, \cdot)$ refers to the local distance used in the point alignment.
The first two lines come from the boundary condition in DTW.
The alignment between the first pair of points and the last pair of points must contribute to the accumulated sum of DTW.
Each point in $Q$ must align with some point in $C$, and vice versa. Each alignment contributes a local distance to the final sum.
The minimum possible local distance for the alignment of $q_{\max}$ would be the one aligned with $c_{\max}$.
The same applies to the last line in the equation.

There is a simplification of $\lbKim$. Only the first and last pair are used in the lower bound computation. In the normalized time series, the third and fourth rows in Equation~\ref{eq:lb-kim} should have small values~\cite{rakthanmanon2013addressing}. Ignoring them can improve the complexity from $\mathcal{O}(n)$ to $\mathcal{O}(1)$.
The simplified lower bound is $\lbKimFL = \operatorname{d}(q_1, c_1) + \operatorname{d}(q_{-1}, c_{-1})$.

\noindent\RuninHead{Keogh Lower Bound~\cite{keogh2005exact}} $\lbKeogh$ builds the upper $U$ and lower envelopes $L$ of one of the time series out of the two compared sequences.
Usually, the envelopes are constructed for $Q$ instead of $C$ as we will compare one $Q$ against many $C$'s. 
Otherwise, we need to build the envelopes for each candidate instead~\cite{rakthanmanon2013addressing}. 

\begin{figure}[tb]
    \centering
    \includegraphics[width=1.0\columnwidth]{figures/lb_keogh-lb_shen-visualization.pdf}
    \caption{Visualization of $\lbKeogh$ and $\lbShen$}
    \label{fig:lb_keogh-lb_shen-visualization}
\end{figure}
To the best of our knowledge, they are the first to interpret the constraint as a restriction on the indices of the warping path $w_k = (i, j)_k$ such that $j - r \leq i \leq j + r$, where $r$ defines the allowable deviation of alignment between $q_i$ and $c_j$.
For ease of exposition, we focus on the most used constraint in the literature, which is the Sakoe-Chiba Band~\cite{tan2018efficient}.
Two sequences are constructed for $Q$, namely the upper $U^Q$ and lower envelopes $L^Q$ of $Q$ as shown in Figure~\ref{fig:lb_keogh-lb_shen-visualization}.
For each $q_i$, we would assign a window of $q_i$ based on its index $i$ as follows.

\begin{equation}
    \label{eq:lb_keogh-envelope}
    \begin{aligned}
        U_i^Q = \max(q_{\max(1,i-r)}:q_{\min(i+r,m)}) \\
        L_i^Q = \min(q_{\max(1,i-r)}:q_{\min(i+r,m)}) \\
    \end{aligned}
\end{equation}

$\max(1, \cdot)$ and $\min(\cdot, m)$ are used for handling the boundary cases.
$U^Q$ and $L^Q$ together form a bounding envelope that encloses the original $Q$, it is the grey region in the figure.
For each $c_j$, either $(c_j, U_j^Q)$ or $(c_j, L_j^Q)$ corresponds to the possible alignment that contributes the minimum distances if $c_j$ falls outside the envelope.
Herein, the lower bound is the sum of these distances, as shown in the following equation.

\begin{equation} 
    \label{eq:lb_keogh}
    \lbKeogh(Q,C) = \sum_{j=1}^{n}
    \begin{cases}
        d(c_j, U_j^Q) & \text{if } c_j > U_j^Q\\
        d(c_j, L_j^Q) & \text{if } c_j < L_j^Q\\
        0 & \text{otherwise}\\    
    \end{cases}
\end{equation}

Visually, these distances are the distances between $c_j$ outside the envelope and the vertically corresponding points on the envelope.
The distances are the green bars in the figure.
Equation~\ref{eq:lb_keogh} returns the sum of the green bars.

\noindent\RuninHead{Shen Lower Bound~\cite{shen2017searching, shen2018accelerating}} It leverages the boundary and continuity conditions to create a lower bound of DTW. 
It can be used to lower-bound the USDTW with slight modification.
The intuition is to find the minimum possible alignment of each $c_j$ with points in $Q$, subject to the local constraint $r$ from DTW and the global constraint $l$ from US.

We will first introduce $\operatorname{LB}_\mathrm{Shen}$ for DTW.
Given the candidate sequence $C$ with length $n$, 
for each $c_j$, we create its possible reach $\mathbbm{q}_j$ in $Q$ under DTW as in Equation~\ref{eq:lb_shen-dtw-q}.
The elements $\mathbbm{q}_j$ form an indexed collection $\mathbbm{Q} = (\mathbbm{q}_1, \mathbbm{q}_2, \dots, \mathbbm{q}_{\min(\floor{lm}, n)}$).

\begin{equation} 
    \label{eq:lb_shen-dtw-q}
    \mathbbm{q}_j = (q_{\max(1, j-r)}, \ldots, q_{\min(j + r, m)})
\end{equation}

We define $\delta(x, Y) = \min_{y \in Y}d(x, y)$.
The possible minimum cost contributed by the alignment of $c_j$ and some point in $Q$ to the accumulated sum would be $\delta(c_j, \mathbbm{q}_j)$.
The lower bound $\lbShen$ is defined as:

\begin{equation}
    \label{eq:lb_shen-dtw}
    \lbShen(Q, C) = \operatorname{d}(c_1, q_1) + \sum_{j=2}^{n-1} \delta(c_j, \mathbbm{q}_j) + \operatorname{d}(c_n, q_m)
\end{equation}

We direct the reader to \cite{shen2017searching} for the formal proof, while an intuition of the proof is presented here.
There are three items on the right-hand side of Equation \ref{eq:lb_shen-dtw}.
The continuity requirement ensures that each $c_i$ is involved in at least one alignment.
The first and the last items are from the boundary condition.
The middle item returns the possible minimum distances contributed by each $c_j$'s, where $2 \leq j \leq n-1$.
It is obvious that $\lbShen$ in Equation~\ref{eq:lb_shen-dtw} is tighter when we use the first pair of points and the last pair of points instead of doing the middle summation all from $j=1$ to $n$.
It is because the distance contributed by the first pair (last pair) must be greater than $\delta(c_1, \mathbbm{q}_1)$ ($\delta(c_n, \mathbbm{q}_n)$).

It is proven that it is tighter than $\lbKeogh$~\cite{shen2017searching}.
It is shown in Figure~\ref{fig:lb_keogh-lb_shen-visualization} visually.
The black bars refer to the additional lower bound distance sum on top of $\lbKeogh$.
%
We will give an intuitive proof here.
Both $\lbKeogh$ and $\lbShen$ compute the lower bounds by summing the local distance resulting from the alignment of each $c_j$ with some points in $Q$, which is guaranteed to be not greater than the partial distance contributed by the real alignment, which we only know until we compute the exact distance measure.
In $\lbKeogh$,  if this $c_j$ falls outside the envelope, this local distance is the vertical distance between $c_j$ and the envelope.
If $c_j$ falls inside, this local distance would be 0.
For those points outside the envelope, the local distances for $c_j$ of $\lbKeogh$ and $\lbShen$ are the same.
But $\lbShen$ aims to return the minimum possible partial distance for each $c_j$, even within the envelope.
Hence, $\lbKeogh \leq \lbShen$.

In order to compute Equation~\ref{eq:lb_shen-dtw} efficiently, we first sort sequences $\mathbbm{q}_j$ and the resulting sorted sequences are denoted as $\tilde{\mathbbm{q}_j}$. 
% The elements $\tilde{\mathbbm{q}_j}$ forms an indexed collection $\tilde{\mathbbm{Q}} = (\tilde{\mathbbm{q}}_1, \tilde{\mathbbm{q}}_2, \dots, \tilde{\mathbbm{q}}_{\min(\floor{lm}, n)}$).
The sorted sequences allow us to do a binary search when we are calculating $\delta(c_j, \tilde{\mathbbm{q}}_j)$ in contrast to $\delta(c_j, \mathbbm{q}_j)$. The sorting only needs to be done once because we are testing the same $Q$ with different candidates.
% To note, for unconstrained DTW (i.e., $r = \infty$), $\mathbbm{q}_j$ is the same for each $i$.
% thus only one sorting is enough.

It can be extended to the USDTW case~\cite{shen2017searching, shen2018accelerating}.
The possible reach now is not only defined by $r$, but also by the scaling factor bound $l$.

\begin{equation}
    \label{eq:lb_shen-usdtw-q}
    \mathbbm{q}_j = (q_{\max(1, \ceil{j/l} - r)}, \ldots, q_{\min(\floor{jl} + r, m)})
\end{equation}

\cite{shen2017searching} proves the following theorem to allow us to consider the lower bound between each prefix of $C$ and $Q$ without the scaling up of each prefix of $C$ (i.e., $C(1:k)^L$) and the scaling up of $Q$ (i.e., $Q^L$).

\begin{theorem}[] 
\label{thm:lb_shen}
    For any $\ceil{m/l} \leq k \leq \min(\floor{lm}, n)$, 
    $\operatorname{DTW}_r(Q^{\min(\floor{lm},n)}, C(1:k)^{\min(\floor{lm},n))})$ is always lower bounded by $\sum_{j=1}^k \delta(c_j, \mathbbm{q}_j)$.
\end{theorem}

Recall that USDTW calculates DTW distances between each rescaled prefix of $C$ to the rescaled $Q$, and outputs the minimum DTW distance, as in Equation~\ref{eq:usdtw}.
The incremental nature of the lower bound frees us to calculate the lower bound of each DTW distance from scratch.
This theorem allows us to first calculate the lower bound of the shortest prefix $C(1:\ceil{m/l})$ of $C$ and $Q$, and then incrementally calculate the lower bound of the longer prefix with length from $\ceil{m/l} + 1$ to $\min(\floor{lm}, n)$ by adding on each $\delta(c_j, \mathbbm{q}_j)$.
To note, it also means that if $\lbShen(Q, C(1:k))$ is larger than a value, namely $\mathrm{bsf}$, $\lbShen(Q, C(1:k'))$, where $k' > k$, would also be larger than $\mathrm{bsf}$.
%
To note, we can tighten $\lbShen$ by using $\operatorname{d}(c_1,q_1)$ instead of $\delta(c_1,\mathbbm{q}_1)$.

The above analysis can also apply to the case of US distance.
We only need to define the corresponding reaches as follows: 
\begin{equation}
    \label{eq:lb_shen-us-q}
    \mathbbm{q}_j = (q_{\max(1, \ceil{j/l})}, \ldots, q_{\min(\floor{jl}, m)})
\end{equation}
% However, it may slow down computation because ED already runs in linear time.
% The reduction in time from pruning unpromising candidates may not compensate for the extra time we spend computing the lower bound.

