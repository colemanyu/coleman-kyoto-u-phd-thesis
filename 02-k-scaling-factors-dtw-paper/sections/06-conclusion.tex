\section{Concluding Remarks} \label{sec:conclusion}

In this study, we proposed a novel distance measure framework, Piecewise Scaling Distance (PSD), which relaxes the strict assumption of a single uniform scaling factor across the entire time series. 
We presented an exact dynamic programming (DP) algorithm to solve this problem.
To enhance efficiency and prevent pathological segment alignments, we introduced a constraint version, which limits the search space of segment lengths based on given scaling factors. 
To enhance computational efficiency, we integrated two optimization techniques for the general PSD framework.
In addition, we propose incorporating a lower-bounding strategy to accelerate PSDTW.
Our experimental results demonstrate the necessity and effectiveness of PSD when identifying matches between a query $Q$ and a candidate $C$ under piecewise scaling distortions.

We outline several directions for future research. 
First, we aim to develop a lower bound specifically optimized for PSED that can improve actual runtime.
Second, while the current PSDTW algorithm requires the number of segments $P$ to be specified as a hyperparameter, it is preferable for the algorithm to determine this value adaptively.
A simple heuristic is to test a range of $P$ values and select the configuration that yields the minimum distance
Finally, we plan to investigate efficiency improvements for PSDTW. 
Currently, PSDTW computes dynamic time warping on two growing subsequences after scaling. While techniques for Incremental DTW~\cite{leodolter2021incdtw} allow for the reuse of the accumulating cost matrix $D$ to avoid redundant calculations, applying this to PSDTW is non-trivial. 
The interpolation performed before DTW fundamentally alters the subsequence structure, preventing the straightforward extension of $D$ (e.g., by appending rows or columns) to reuse the computed $D$ that is possible in standard DTW.