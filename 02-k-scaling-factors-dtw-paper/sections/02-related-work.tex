\section{Related Work} \label{sec:related-work}

This study focuses on distance measures of time series.
For the overall review of time series, we direct the readers to \cite{fu2011review, esling2012timeseries} for a more comprehensive understanding of this field.

% Distance measure
For many tasks, having appropriate distance measures that align with our intuition for the domains we work with is essential.
One well-known distance measure is Dynamic Time Warping (DTW).
It is initially designed for speech analysis~\cite{sakoe1978dynamic}.
However, DTW is computationally expensive.
Lower bounds are used to speed up time series similarity search by admissibly pruning the unpromising candidates.
One of the popular exact lower bounds of DTW is $\lbKeogh$.
\cite{rakthanmanon2013addressing} improves the scalability of DTW by introducing a subsequence search suite of their four novel ideas, namely the UCR suite.
% It successfully re-establishes DTW as a practical tool rather than a theoretical study. 
For an overall review of lower bounds, we refer readers to \cite{tan2019elastic, shifaz2023elastic}.
There is an approximate algorithm that approximates DTW with high accuracy while drastically cutting down the time and space requirements~\cite{salvador2007accurate}.

While ED is sensitive to distortions in the time axis, uniform scaling (US) has been shown to be a critical invariance in domains such as motion capture.
\cite{keogh2004indexing} demonstrated that DTW is insufficient for handling global scaling effects, and that identifying DTW is not the solution to achieve this kind of invariance. There is a need for US.
\cite{yankov2007detecting} extends the importance of uniform scaling to motif discovery. The authors show that meaningful motifs often suffer from a global scaling effect, causing standard motif finding algorithms to miss them completely.

To the best of our knowledge, three studies analyze the combination of US and DTW, namely USDTW.
It was first proposed by~\cite{fu2008scaling}.
It extended $\lbKeogh$ to bound the USDTW.
However, the extended $\lbKeogh$ is still too loose with invariance to large amounts of uniform scaling.
\cite{shen2017searching} and its follow-up study~\cite{shen2018accelerating} proposed a new lower bound, namely $\lbShen$
~\footnote{It is denoted as $\lbNew$ in the original study. We rename it to prevent any confusion.}
, which has been shown to be tighter than $\lbKeogh$ on USDTW.

To our surprise, despite a fruitful discussion of DTW, US, and USDTW, no study has proposed a distance measure capable of handling scaling effects across multiple scaling factors.
This is precisely what we will address in this study.