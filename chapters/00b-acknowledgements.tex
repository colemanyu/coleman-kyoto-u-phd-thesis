\chapter*{Acknowledgements}

To begin, I want to express my deepest gratitude to my PhD supervisor at Kyoto University, Prof. Tatsuya Akutsu. 
I thank him for his kindness and patience. 
He is a brilliant researcher who can explain complex concepts simply.
In a seminar, he showed that simply using clear notation makes presentations much easier to follow.
More importantly, he helps students when they are in ``valleys'' in their lives.
When a car is off track, it needs a push—so do people.
%
I am equally indebted to my MPhil supervisor at The Hong Kong University of Science and Technology (HKUST), Prof. Raymond Chi-Wing Wong. He has continued to help and advise me even after I graduated from HKUST. 

Akutsu-sensei taught me the importance of reading good materials.
In the weekly seminars, he organized reading seminars, assigning some of his favorite books for us to read and discuss.
His favorite book is ``Probability and Computing'' by Mitzenmacher and Upfal.
Additionally, through the journal club he organized, students were asked to present the core ideas from their ten selected papers published in leading venues, such as Nature, Science, Cell, and PNAS.
Reading beyond my fields helped me identify research gaps others missed.
I think some great ideas are born when we merge the ideas from different fields.
He also emphasized the importance of mathematics and proof. 
It is what computer scientists can stand out in the field of bioinformatics.
I think it becomes especially true now, since advances in AI make it possible for almost everyone to code (i.e., Vibe coding).
He advised me to use the existing tools, libraries, and studies to leverage my own research.
He also taught me that a good researcher must master both oral and written presentation; otherwise, others may misinterpret your talent and the effort you put in.
When we are reading a paper, he urges us to identify the novelty and analyze the pros and cons of the paper.

Raymond taught me the power of focus and ``First Principles''. 
I was always surprised that he did not use reference management software like Zotero, preferring to annotate hard copies and even type .bib files manually. 
He told me that when he starts to do research, he will print out the papers and get focused on the stack of papers (hard copies) in front of him.
I am not saying it is beneficial not to use tools, but I want to emphasize the power of focus that underlies his work routine.
He taught me that every good research starts with a set of good papers.
He also emphasized that there is no right or wrong in research, only what you choose to do about it.
His research receipt works as follows.
When you are tackling a problem, you first review the relevant existing studies to understand the state of the art (SOTA) for it.
If the existing work addresses your particular problem, you can apply and adapt it to your problem setting.
But most of the time, since your problem must be a particular version of a general problem, the existing general solution should not work well for it. 
It means that you have some room to improve it. And this is the research gap!.
It reminds me of when we deal with the NP-complete problem.
As Kleinberg and Tardos's Algorithm Design (Chapter 10) suggests, an NP-complete problem (assuming $P \neq NP$) does not allow us to have an algorithm that possesses all three of the following desired properties simultaneously: Efficiency, Correctness, and Generalization.
Hence, it is sometimes preferable to address a specific instance of the problem rather than the general one.
He also emphasized the theory. He consistently noted that you need to add theory to the paper to strengthen it.
He always mentioned that you need three motivations (why you are doing it) and three contributions (what you have done).
Sometimes, I ask why he can write so fast, and he replies, ``If you know what you are doing, then you write fast.''.
He also taught me that when you don't know some of the ideas, you just go back to the original idea. Doing a ``deep first search’’ can help you reach the first principle, a concept that Elon Musk, one of the greatest entrepreneurs of our time, always emphasized.

Without these two supervisors, I would not have made it this far. I could not have finished this program.

I would like to take this opportunity to share some of my thoughts on my PhD journey.
I hope the readers may learn something from what I have gone through.
My PhD journey was, to quote Dickens, ``the best of times and the worst of times'', and to quote Churchill, ``This was their finest hour''. 
For the best parts, it allows me to explore both in my daily life and in my research.
I tried many things, walked many roads, drank many colas and alcohol, and met many people. 
This helped me see problems from new perspectives.
Regarding the worst parts, I am reminded of a quote from one of my favorite movies, ``Les Choristes'': ``Fond de l'étang''. It literally means ``Bottom of the Pond''. 
At times, I felt like a frog at the well's bottom, trying hard to get out. 
Research is fun but also hard.
Research is about exploring something new. 
It is about publishing (so others can learn from it).
When I am stuck, the best solution is to aim for a reachable, well-defined goal.
The goal should be clear, with obvious rewards and requirements.
Also, make sure the effort of your actions can be accumulated.
Like the frog, do not jump randomly, but aim to move to stable platforms towards escape.
Then, each jump matters for your progress.
Two of the materials helped me; they are ``Eat the Frog!'' and ``THE PH.D. GRIND''.

I would like to thank my thesis committee members,
Prof. Tatsuya Kawahara and Prof. Hisashi Kashima.
I appreciate their willingness to serve as committee members, to provide me with valuable advice.
%
I would like to acknowledge Tamura-sensei and Mori-sensei (a former member) of the Akutsu laboratory. Tamura-sensei taught me to focus on the input-output relationship when encountering new algorithms/methods, and Mori-sensei introduced me to the fascinating application of time-series analysis to gene expression data, especially in trajectory inferences (i.e., pseudotime).
%
I would like to thank the many communities that made my life in Japan colorful.
%
Thank you to the people of the dormitories where I lived at the Uji and Yoshida campuses. 
%
I am grateful to my Japanese language teachers. 
%
I also cherish the friends I met through Akutsu's laboratory and the extracurricular activities, including Kendo, Table Tennis, Naginata, Karate, and Aikido.
I also met many interesting people from various international student events.

I am honored to receive the Asian Future Leaders Scholarship Program (AFLSP) scholarship to support my studies in Japan. 
I have made a great choice by enrolling in Kyoto University Design School, which has provided me with many valuable interdisciplinary experiences and opportunities to meet people outside my field. In this program, I have conducted fieldwork in Okinawa, Hong Kong, and Bali, a rare opportunity for researchers in computer science. 
I would also like to thank the Institute for Chemical Research (ICR) for the Research Assistant position in the Akutsu laboratory.

Finally, I would like to express my deepest gratitude to my family and my friends who have stayed by my side with their countless acts of support; I have nothing to return to them but my love and time. 

A special acknowledgment goes to my niece. As you grow up, I hope you explore the world and fulfill your eagerness for knowledge. Find materials that interest you. One day, you might find this thesis online and, I hope, find it worth reading and inspiring.