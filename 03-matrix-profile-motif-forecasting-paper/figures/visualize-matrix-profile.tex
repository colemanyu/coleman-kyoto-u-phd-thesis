\begin{figure}[htbp]
\centerline{\includegraphics[width=1.0\columnwidth]{figures/visualize_matrix_profile.pptx.pdf}}
\caption{The \textcolor{tab:green}{left} (\textcolor{tab:blue}{right}) matrix profile and the \textcolor{tab:red}{matrix profile} of a time series. The \textcolor{tab:red}{matrix profile} shows the distances between each $m$-subsequence and its nearest neighbor, where $m$ is a user-given value.
The \textcolor{tab:green}{left} (\textcolor{tab:blue}{right}) matrix profile shows the same information but is limited to its left (right) nearest neighbor.
For a particular $m$-subsequence shown by the right gray box, its nearest neighbor is the left gray box, as indicated by the dashed line in the \textcolor{tab:green}{left matrix profile} and the \textcolor{tab:red}{matrix profile}.
Similarly, the nearest neighbor of the left gray box is the right gray box, as indicated by the dashed line in the \textcolor{tab:blue}{right matrix profile} and the \textcolor{tab:red}{matrix profile}.
The first (last) point in each box is denoted by a \textcolor{tab:red}{red} circle (triangle).
$h$ denotes the length of the immediate subsequence of the nearest neighbor.
Intuitively, this subsequence should be similar to the immediate subsequence (i.e., future) of the right gray box.
To note, the \textcolor{tab:green}{left} (\textcolor{tab:blue}{right}) matrix profile starts (ends) at a later (earlier) index because the corresponding nearest neighbor with length $m$ does not exist.
}
\label{fig:visualize-matrix-profile}
\end{figure}