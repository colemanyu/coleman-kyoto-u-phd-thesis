\section{Introduction} \label{sec:introduction}

Humans have made predictions since ancient times.
% Ancient times
In ancient societies, accurate predictions were important to the success of subsistence activities such as hunting, planting, and harvesting.
There was a need to predict weather dynamics, such as rainfall and temperature.
For example, it is crucial to plant during the period with sufficient rainfall and appropriate temperatures.
Our ancestors used divination tools such as turtle shells, wooden blocks (moon blocks), or bones to make predictions.
Without doubt, the accuracy was not guaranteed.
% In the past, more advanced predictions have involved astronomy. They focused on predicting celestial events, such as eclipses, for religious and agricultural purposes.
% Modern times
Even in modern societies, prediction is still essential.
Predicting traffic-jam patterns only a few hours ahead can save time by enabling the selection of an alternative route~\cite{lai2018modeling}.
A wealth could be created by forecasting stock market trends.
Predicting the future, also known as time series forecasting, is crucial.

% \cite{plageras2018efficient}
With advances in hardware technologies, we collect enormous amounts of data from diverse sources, such as smart sensors and social media platforms, continuously in the form of time series data.
%
A time series is an ordered sequence of measures, represented in real-valued numbers, at discrete equal-interval timestamps~\cite{chatfield2003analysis}.
%
The vast data collections have created the era of ``Big Data'', which provides a wealth of datasets for developing and deploying reliable, robust, data-driven forecasting techniques to discover patterns and extract valuable information~\cite{al-jarrah2015efficient}.
%
Applications can be found in the financial sector, such as predicting business cycles and stock market movements~\cite{li2024deep, lai2018modeling, sapankevych2009time, chen2023long} and the medical field, such as the status of critical patients according to their vital signs~\cite{torres2021deep, dasilva2021deepsigns} and the propagation of diseases such as influenza~\cite{chen2023long, wu2020deep} and COVID-19~\cite{torres2021deep, martinez-alvarez2020coronavirus}. 

Many methods have been developed for time series forecasting.
Traditional methods include rolling averages (RA), vector auto-regression (VAR)~\cite{lai2018modeling, box2015time}, and auto-regressive integrated moving averages (ARIMA)~\cite{elsayed2021we, box1970distribution, box2015time}.
Because of their rigorous statistical properties, they have long been the standard.
The shortcomings of ARIMA and its variants include their high computational cost~\cite{lai2018modeling}. 
In contrast, VAR is arguably the most widely used method, particularly in multivariate time series analysis, owing to its simplicity.
However, most of these traditional approaches have certain limitations.
%
% https://medium.com/data-science/understanding-the-limitations-of-arima-forecasting-899f8d8e5cf3
% https://stats.stackexchange.com/questions/558252/what-are-the-downsides-of-arima-models
They perform well when the data meet specific statistical assumptions, such as stationarity~\cite{lim2021timeseries}, which means that the mean and variance of the time series remain constant over time.
% For example, ARIMA performs well only if the underlying stochastic process of the time series satisfies the assumptions of ARIMA.
%
It motivates the community to develop machine learning methods, particularly deep learning methods for time series forecasting. 
Many deep learning models have been proposed, including RNN-based models, CNN-based models, GNN-based models, Transformer-based models, and compound models that incorporate different base models mentioned before~\cite{chen2023long}.
The compound models are promising.
For example, RNNs are well suited to capturing long-term dependencies, whereas CNNs are well suited to capturing short-term dependencies. 
A good way to improve performance is to compound them.
For example, LSTnet~\cite{lai2018modeling} integrates CNN, RNN, and autoregressive~\cite{yule1927method} techniques to extract both short-term and long-term patterns.
Using the occupancy rate of a freeway as an example~\cite{lai2018modeling} to explain these two patterns, the ``short-term'' patterns refer to the morning peaks against evening peaks, while the ``long-term'' patterns refer to the workday patterns against weekend patterns.
Clearly, a good forecaster needs to capture and distinguish both kinds of patterns.
Despite the superior performance deep learning methods have achieved, they tend to be overly complex, opaque, and incur high computational costs compared to traditional techniques.

\begin{figure}[htbp]
\centerline{\includegraphics[width=1.0\columnwidth]{figures/flattening.pptx.pdf}}
\caption{Flattening.}
\label{fig:flattening}
\end{figure}
A recent study~\cite{middlehurst2024bake} shows that, in time series classification, a non-parametric, instance-based method, namely nearest-neighbor classifiers (1-NN) and its generalized form $k$-NN, with appropriate distance measures, such as Dynamic Time Warping (DTW), despite their simplicity, perform well and are therefore commonly used as benchmarks.
In detail, when a new instance is to be classified, $k$-NN finds its $k$ nearest neighbors in the training set and returns their majority label among them.
$k$-NN is considered a lazy learner because the training steps involve only memorizing all the instances verbatim; no higher-level concepts have been learned.
%
In addition, in time series forecasting, a recent study demonstrates that a well-known machine learning baseline, Gradient Boosting Regression Tree (GBRT), such as XGBoost, equipped with an appropriate data engineering of the data, can achieve competitive or even superior performance than the deep learning method.
In detail, they transform the time series forecasting task into a window-based regression problem, as shown in Figure~\ref{fig:flattening}.
For each training window of length $w$ with the last time point $y_i$, $y_i$ and its lagged values $y_{i-1}, y_{i-2}, \dots, y_{i-w+1}$ are concatenated with covariates $x_i^1, x_i^2, \dots, x_i^M$ to form a \textit{predictor} for a multi-output GBRT. This transformation is called flattening.
The corresponding response is the following $h$ points of $y_i$.
It provides a simple, more efficient yet accurate method for time series forecasting.

\begin{figure}[htbp]
\centerline{\includegraphics[width=1.0\columnwidth]{figures/visualize_matrix_profile.pptx.pdf}}
\caption{The \textcolor{tab:green}{left} (\textcolor{tab:blue}{right}) matrix profile and the \textcolor{tab:red}{matrix profile} of a time series. The \textcolor{tab:red}{matrix profile} shows the distances between each $m$-subsequence and its nearest neighbor, where $m$ is a user-given value.
The \textcolor{tab:green}{left} (\textcolor{tab:blue}{right}) matrix profile shows the same information but is limited to its left (right) nearest neighbor.
For a particular $m$-subsequence shown by the right gray box, its nearest neighbor is the left gray box, as indicated by the dashed line in the \textcolor{tab:green}{left matrix profile} and the \textcolor{tab:red}{matrix profile}.
Similarly, the nearest neighbor of the left gray box is the right gray box, as indicated by the dashed line in the \textcolor{tab:blue}{right matrix profile} and the \textcolor{tab:red}{matrix profile}.
The first (last) point in each box is denoted by a \textcolor{tab:red}{red} circle (triangle).
$h$ denotes the length of the immediate subsequence of the nearest neighbor.
Intuitively, this subsequence should be similar to the immediate subsequence (i.e., future) of the right gray box.
To note, the \textcolor{tab:green}{left} (\textcolor{tab:blue}{right}) matrix profile starts (ends) at a later (earlier) index because the corresponding nearest neighbor with length $m$ does not exist.
}
\label{fig:visualize-matrix-profile}
\end{figure}
Moreover, $k$-NN has also shown to be a promising method for time series forecasting~\cite{martinez2019methodology}.
The $k$-NN uses the lagged values of the last time point to form a query $Q$.
It identifies the $k$ previous similar subsequences to $Q$ and uses their immediate subsequences to predict the immediate subsequence, which is the forecasting window, of $Q$.
The intuition is that history repeats itself.
The previous (historical) subsequences that are similar to $Q$ can provide a hint about the future of $Q$. 
They are similar, and so are their immediate subsequences.
%
Figure~\ref{fig:visualize-matrix-profile} depicts this idea. Observe that the immediate subsequence of the right gray box is similar to that of the left gray box.
The left gray box is the nearest neighbor of the right gray in the ``past''.

\begin{algorithm}[t]
	\caption{The brute force approach to compute Matrix Profile}
	\label{alg:mp-brute-force}
	\begin{algorithmic}[1]		
		\For{$i \gets 1$ \textbf{to} $n-m+1$}
        \For{$j \gets 1$ \textbf{to} $n-m+1$}
        \State Compute the z-normalized Euclidean distance between $t_{i, m}$ annd $t_{j, m}$.
        \EndFor
        \EndFor
	\end{algorithmic}
\end{algorithm}
Based on these findings, this study proposes a method to improve the performance of existing forecasters by leveraging information from the nearest neighbors of each subsequence in the target variable.
For each time point $y_i$ of the target variable $Y$, a window of length $w$ is constructed with $y_i$ as the last point, then we retrieve the window's historical nearest neighbors and use their information to create new covariates for the window.
The information includes the similarities between the window and its nearest neighbors, as well as the immediate subsequences of them.
The similarity can be interpreted as a measure of confidence or weight in using the information from the corresponding nearest neighbor.
The intuition is that, if the similarity of the window and a neighbor is high, then the future (i.e., immediate subsequence) of the neighbor should also be similar to the future of the window. 
%
The fundamental difference between this study and previous approaches~\cite{tajmouati2024applying, martinez2019methodology, thi2021forecast} is that they directly use the subsequent points for prediction, whereas we use the nearest neighbor information for each subsequence as covariates, which are used as primitives for other forecasters.
We explain this subtle difference by Figure~\ref{fig:visualize-matrix-profile}.
The previous approaches simply use the information of the nearest neighbors of the last look-back window, which consisted of the last time point and its lagged values (i.e., the right gray box), for prediction.
In contrast, we use the information of the nearest neighbors of \textbf{all} of the windows.

We use the Matrix Profile~\cite{yeh2016matrix, zhu2017matrix} to annotate the nearest neighbor for each $m$-subsequence of a time series $T$ of length $n$.
The distance is the z-normalized Euclidean distance.
It may seem computationally expensive to perform this annotation at first glance.
%
Algorithm~\ref{alg:mp-brute-force} shows the brute force approach to compute the matrix profile.
The three for loops indicate that the computational complexity is $\mathcal{O}(n^2m)$.
The space complexity is $\mathcal{O}(n^2)$ because of the pairwise distance of each subsequence with the other subsequence.
%
However, the matrix profile can be computed in $\mathcal{O}(n^2)$ using an exact method, namely STOMP~\cite{yeh2016matrix} or its community-open-sourced version, STUMP~\cite{law2019stumpy}.
Besides, the running time can be further sped up by parallelization for a single machine with multiple computation units, such as CPUs or GPUs.
The tool also allows us to compute the left matrix profile to find the left nearest neighbor of each window.
To note, the matrix profile annotates a time series with information about the nearest neighbor of each subsequence, including the similarity with its nearest neighbor and its location, as shown in Figure~\ref{fig:visualize-matrix-profile}.

In this study, we make the following contributions:
\begin{itemize}
  \item We are the first to propose leveraging the matrix profile to create meaningful covariates that improve forecaster performance. 
\end{itemize}

The rest of this paper is organized as follows.
In Section~\ref{sec:method}, we introduce the necessary background knowledge, then introduce our method.
Section~\ref{sec:experiments} contains an empirical evaluation.
Finally, we conclude this paper and provide future work in Section~\ref{sec:conclusion}.