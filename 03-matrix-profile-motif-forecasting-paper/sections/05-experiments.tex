\section{Experiments} \label{sec:experiments}

\begin{table}[htbp]
\centering
\caption{Dataset Statistics. 
$N$ is the number of time series in the dataset, while $|T| = n$ is the length of each time series. 
``rate'' refers to the measuring rate.
$w$ is the size of the look-back window.
$h$ is the size of the forecasting window, also known as the forecasting horizon.
$T_{\mathrm{train}}$ is the training subsequence of $T$.
$T_{\mathrm{test}}$ is the test subsequence of $T$.
To note, $|T_{\mathrm{train}}| + |T_{\mathrm{test}}| = n$.
}
\label{tab:dataset_statistics}
% \resizebox{\textwidth}{!}{
\begin{tabular}{lrrrrrrrr}
\toprule
 & \multicolumn{3}{c}{\textbf{Data}} & \multicolumn{5}{c}{\textbf{Forecasting Task}} \\
\cmidrule(lr){2-4} \cmidrule(lr){5-9}
\textbf{Dataset} & $N$ & $n$ & rate & $w, h$ & $|T_{\mathrm{train}}|$ & $|T_{\mathrm{test}}|$ \\
\midrule
Electricity~\cite{sen2019think} & 70 & 26,136 & hourly & 24 & 25,968 & 168 \\
Traffic~\cite{sen2019think} & 90 & 10,560 & hourly & 24 & 10,392 & 168 \\
PeMSD7(M)~\cite{sen2019think} & 228 & 12,672 & /5 mins & 9 & 11,232 & 1,440 \\
Exchange-Rate~\cite{lai2018modeling} & 8 & 7,536 & daily & 24 & 6,048 & 1,488 \\
\bottomrule
\end{tabular}
% }
\end{table}
The code and data are available at \url{https://github.com/colemanyu/matrix-profile-motif-forecasting}.
To note, for all these datasets, we do not have $x_t$. In other words, we lack auxiliary channels.
However, each time point contains a timestamp. From it, we can infer the covariates associated with the timestamp.
We refer to them as the original covariates.

In this study, we focus on univariate time series forecasting.
There is only one single target variable $Y$.
We predict the future of $Y$ only based on the past of it.
The dataset used is listed in Table~\ref{tab:dataset_statistics}.
For each time series $T$ in the dataset, it is split into two contiguous time series, namely $T_\mathrm{train} = T(1:\mathrm{split})$ and $ T_\mathrm{test} = T(\mathrm{split+1:|T|})$, where $\mathrm{split}$ defines the training-test split, which is shown in the column 6 in Table~\ref{tab:dataset_statistics}.
To note, the $N$ time series in a dataset are considered as multiple, independent unvariate time series instead of a multivaraite time series with channels $= N $.
The settings of $w$ and $h$ are adopted from the original corresponding papers~\cite{elsayed2021we}.
It is not necessary for them to be the same.
The original papers of the comparison methods use this setting.
So, we use the same setting for fair comparison.

The data used includes the target values in the form of lagged values, and its covariates.

GBRT-NN refers to a model that incorporates immediate sequence information.
GBRT-NN-S refers to a model that incorporates immediate sequence information with the similarity information.
The experimental results are shown in Table~\ref{tab:results_without_covariates} and Table~\ref{tab:results_with_covariates}.

As shown in Table~\ref{tab:results_without_covariates}, in the case of working without original covariates, our methods can improve the performance among all three metrics, except in the case of RMSE in the Electricity dataset.

As shown in Table~\ref{tab:results_with_covariates}, in the case of working with original covariates, our methods can improve the performance among all three metrics, except in the case of RMSE in the Traffic dataset.

We note that in the Exchange-Rate dataset, our methods perform similarly to the original method GBRT.
We argue that this is because, in this dataset, it depends solely on the look-back window and not on any far-away historical events.

\begin{table}[htbp]
    \centering
    \caption{Experimental Results without original covariates (bold represents the best result and underlined represents the second best) (* refers to the results reported from~\cite{elsayed2021we}).}
    \label{tab:results_without_covariates}
    \resizebox{\textwidth}{!}{%
    \begin{tabular}{llccccccc}
        \toprule
        \textbf{Dataset} & \textbf{Metric} & \textbf{LSTNet}*~\cite{lai2018modeling} & \textbf{TRMF}*~\cite{yu2016temporal} & \textbf{DARNN}*~\cite{qin2017dualstage} & \textbf{ARIMA}*~\cite{makridakis1997arma} & \textbf{GBRT}~\cite{elsayed2021we} & \textbf{GBRT-NN} & \textbf{GBRT-NN-S} \\
        \midrule
        \multirow{3}{*}{Electricity~\cite{sen2019think}} 
        & RMSE & 1095.309 & \underline{136.400} & 404.056 & 181.210 & \textbf{136.254} & 142.035 & 138.354 \\
        & WAPE & 0.997 & \textbf{0.095} & 0.343 & 0.310 & 0.103 & 0.101 & \underline{0.100} \\
        & MAE & 474.845 & \textbf{53.250}& 194.449 & 154.390 & 57.929 & 56.464 & \underline{55.972} \\
        \midrule
        \multirow{3}{*}{Traffic~\cite{sen2019think}} 
        & RMSE & 0.042 & 0.023 & 0.015 & 0.044  & \underline{0.012} & 0.016 & \textbf{0.008} \\
        & WAPE & 0.102 & 0.161 & 0.132 & 0.594 & 0.108 & \underline{0.079} & \textbf{0.037} \\
        & MAE & 0.014 & 0.009 & 0.007 & 0.032 & 0.006 & \underline{0.004} & \textbf{0.002} \\
        \midrule
        \multirow{3}{*}{PeMSD7~\cite{sen2019think}} 
        & RMSE & 55.405 & \textbf{5.462} & 5.983 & 15.357 & 5.610 & 5.575 & \underline{5.557} \\
        & WAPE & 0.981 & \underline{0.057} & 0.060 & 0.183 & \textbf{0.051} & \textbf{0.051} & \textbf{0.051} \\
        & MAE & 53.336 & 3.329 & 3.526 & 10.304 & 2.984 & \underline{2.965} & \textbf{2.957} \\
        \midrule
        \multirow{3}{*}{Exchange-Rate~\cite{lai2018modeling}} 
        & RMSE & \textbf{0.018} & \textbf{0.018} & 0.025 & 0.123 & 0.020 & \underline{0.019} & \underline{0.019} \\
        & WAPE & 0.017 & \textbf{0.015} & 0.022 & 0.170 & \underline{0.016} & \textbf{0.015} & \textbf{0.015} \\
        & MAE & 0.013 & \textbf{0.011} & 0.016 & 0.101 & \underline{0.012} & \underline{0.011} & \underline{0.012} \\
        \bottomrule
    \end{tabular}
    }
\end{table}
% \underline \textbf
\begin{table}[htbp]
    \centering
    \caption{Experimental Results with original covariates (bold represents the best result and underlined represents the second best) (* refers to the results reported from~\cite{elsayed2021we}).}
    \label{tab:results_with_covariates}
    \resizebox{\textwidth}{!}{%
    \begin{tabular}{llccccccc}
        \toprule
        \textbf{Dataset} & \textbf{Metric} & \textbf{DeepGlo}*~\cite{sen2019think} &  \textbf{GBRT}~\cite{elsayed2021we} & \textbf{GBRT-NN} & \textbf{GBRT-NN-S} \\
        \midrule
        \multirow{3}{*}{Electricity~\cite{sen2019think}} 
        & RMSE & 141.285  & 132.669 & \textbf{123.591} & \underline{124.215} \\
        & WAPE & 0.094 & 0.0929 & \textbf{0.089} & \textbf{0.089} \\
        & MAE & 53.036 & 52.0232 & \textbf{49.606} & \underline{49.739} \\
        \midrule
        \multirow{3}{*}{Traffic~\cite{sen2019think}} 
        & RMSE & 0.026  & \underline{0.014} & 0.018 & \textbf{0.009} \\
        & WAPE & 0.239 & 0.109 & \underline{0.101} & \textbf{0.062} \\
        & MAE & 0.013 & 0.006 & \underline{0.006} & \textbf{0.003} \\
        \midrule
        \multirow{3}{*}{PeMSD7~\cite{sen2019think}} 
        & RMSE & 6.490  & 5.191 & \underline{5.163} & \textbf{5.156} \\
        & WAPE & \underline{0.070} & \textbf{0.048} & \textbf{0.048}& \textbf{0.048} \\
        & MAE & 3.530 & 2.796 & \underline{2.779} & \textbf{2.778} \\
        \midrule
        \multirow{3}{*}{Exchange-Rate~\cite{lai2018modeling}} 
        & RMSE & 0.038  & \underline{0.020} & \textbf{0.019} & \textbf{0.019} \\
        & WAPE & 0.038 & \underline{0.016} & \textbf{0.015} & \textbf{0.015} \\
        & MAE & \underline{0.029} & \textbf{0.012} & \textbf{0.012} & \textbf{0.012} \\
        \bottomrule
    \end{tabular}
    }
\end{table}

