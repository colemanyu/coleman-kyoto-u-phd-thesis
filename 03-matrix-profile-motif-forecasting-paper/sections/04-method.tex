\section{Method} \label{sec:method}
%%%
In this section, we first formulate the time series forecasting problem and then discuss how to leverage motifs to improve the performance of the Gradient Boosting Regression Tree (GBRT), our forecaster, in the following part.
In that part, we present how the GBRT can be used in this problem, followed by the generation of motifs by a time series primitive, namely Matrix Profile. 

\subsection{Problem Formulation}
\subsection{Gradient Boosting Regression Tree (GBRT)}
\subsection{Matrix Profile}
\subsection{Evaluation Metrics}
% \cite{li2024deep}, \cite{lai2018modeling}
% https://robjhyndman.com/hyndsight/wape.html

We used the following three evaluation metrics defined as:
\begin{itemize}
    \item Root Mean Square Error (RMSE)
    \begin{equation} 
        \label{eq:rmse}
        \operatorname{RMSE} = \sqrt{\frac{1}{n}\sum_{i=1}^n (y_i-\hat{y_i})^2}
    \end{equation}
    \item Weighted Absolute Percentage Error (WAPE)
    \begin{equation} 
        \label{eq:wape}
        \operatorname{WAPE} = \frac{\sum_{i=1}^n |y_i-\hat{y_i}|}{\sum_{i=1}^n |y_i|}
    \end{equation}
    \item Mean Absolute Error (MAE)
    \begin{equation} 
        \label{eq:mae}
        \operatorname{MAE} = \frac{1}{n}\sum_{i=1}^n |y_i-\hat{y_i}|
    \end{equation}
\end{itemize}
where $n$ is the length of the time series, $y_i$, $\hat{y_i}$ is ground true value and predicted value, respectively.
RMSE and MAE are widely used metrics.
MAE can better reflect the actual error situation than RMSE~\cite{li2024deep}.
WAPE was introduced by \cite{kolassa2007advantages}.
By rewriting Equation~\ref{eq:wape} to Equation~\ref{eq:wape-2}, it is more obvious that it is a weighted absolute percentage error.
\begin{equation} 
    \label{eq:wape-2}
    \operatorname{WAPE} = \sum_{i=1}^n w_i \frac{|y_i - \hat{y_i}|}{|y_i|}
\end{equation}
where the weights are given by
\begin{equation} 
    w_i = \frac{|y_i|}{\sum_{i=1}^n |y_i|}
\end{equation}
For all of them, a lower value is better.

%%%
% time series
% subsequence
% Definition of time series forecasting
% autocorrelation
% single step vs multi step
% How to define "good" and evaluate the method?
% Rolling
% Teacher forcing
% lag feature
% xgboost
% matrix profile
% Exclusion zone
% motif, top-k
% subsequence point, and how to use them, distance (importance), how about idx
% Or using feature importance 
% autocorrelation curve
% Figure matrix profile, distance profile
% Figure k-NN
% Three method (Normal MP, Motif with minimum requirement, Motif family)
% The latter allows us to have a brunch of candidates two allow us to prune the outlier

%%%
\cite[chapter 3]{zhu2018matrix, zhu2017matrix}, 

\begin{definition}[Time series]
    A \textit{time series} $T = t_1, t_2, \dots, t_n$ is a sequence of real-valued numbers with length = $n$.
\end{definition}

\begin{definition}[Subsequence]
    A \textit{subsequence} $T_{i, m} = t_i, t_{i+1}, \dots, t_{i+m-1}$ of a $T$ is a sequence that consists of a continuous subset of the entries from $T$ of length $m$ starting from $i$.
\end{definition}
A sequence with length $= m$ is denoted as $m$-sequence.

\begin{definition}[Distance profile]
    A \textit{distance profile} $D_i = d_{i, 1}, d_{i, 2}, \dots, d_{i, n-m+1}$ of a $T$ is a vector of the Euclidean distances between a given subsequence $T_{i, m}$ and each subsequences in $T$, where $d_{i, j}$ is the distance between $T_{i, m}$ and $T_{j, m}$, $1 \leq i, j \leq n-m+1$.
\end{definition}
The distances are measured between z-normalized time series.


\begin{definition}[Matrix profile]
    A \textit{matrix profile} $P = \min(D_1), \min(D_2), \dots, $\\$\min(D_{n-m+1})$ of $T$ is a vector of Euclidean distances between every subsequence $T_{i, m}$ of $T$ and its nearest neighbor in $T$.
\end{definition}

\begin{definition}[Matrix profile index]
    A \textit{matrix profile index} $I = I_1, I_2, \dots, I_{n-m+1}$ of $T$ is a vector of integers, where $I_i = j$ if $d_{i, j} = \min{D_i}$.
\end{definition}

\begin{definition}[Left distance profile]
    A \textit{left distance profile} $D_i^L = d_{i, 1}, d_{i, 2}, \dots,\allowbreak d_{i, i - \ceil{m/4} -1}$ of $T$ is a vector of Euclidean distances between a given subsequence $T_{i, m}$ and each subsequence that appears before $T_{i, m}$.
    To note, $i - \ceil{m/4} - 1$ is the index location of the last eligible subsequence before $T_{i, m}$ because of the exclusion zone.
\end{definition}

\begin{definition}[Left matrix profile]
    A \textit{left matrix profile} $P^L = \min(D_1^L), \min(D_2^L), \allowbreak  \dots, \min(D_{n-m+1}^L)$ of $T$ is a vector of Euclidean distances between every subsequence $T_{i, m}$ of $T$ and its nearest neighbor in $T$ before it.
\end{definition}

\begin{definition}[Left matrix profile index]
    A \textit{left matrix profile index} $I^L = I_1^L, I_2^L, \dots, I_{n-m+1}^L$ of $T$ is a vector of integers, where $I_i^L = j$ if $d_{i, j} = \min{D_i^L}$.
\end{definition}
