\section{Related Work} \label{sec:related-work}

Many methods have been developed for time series forecasting.
Traditional methods include rolling averages (RA), vector auto-regression (VAR)~\cite{lai2018modeling, box2015time}, and auto-regressive integrated moving averages (ARIMA)~\cite{elsayed2021we, box1970distribution, box2015time}.
Because of their rigorous statistical properties, they have long been the standard.
The shortcomings of ARIMA and its variants include their high computational cost~\cite{lai2018modeling}. 
In contrast, VAR is arguably the most widely used method, particularly in multivariate time series analysis, owing to its simplicity.
However, most of these traditional approaches have certain limitations.
%
% https://medium.com/data-science/understanding-the-limitations-of-arima-forecasting-899f8d8e5cf3
% https://stats.stackexchange.com/questions/558252/what-are-the-downsides-of-arima-models
They perform well when the data meet specific statistical assumptions, such as stationarity~\cite{lim2021timeseries}, which means that the mean and variance of the time series remain constant over time.
% For example, ARIMA performs well only if the underlying stochastic process of the time series satisfies the assumptions of ARIMA.
%
It motivates the community to develop machine learning methods, particularly deep learning methods for time series forecasting. 
Many deep learning models have been proposed, including RNN-based models, CNN-based models, GNN-based models, Transformer-based models, and compound models that incorporate different base models mentioned before~\cite{chen2023long}.
The compound models are promising.
For example, RNNs are well suited to capturing long-term dependencies, whereas CNNs are well suited to capturing short-term dependencies. 
A good way to improve performance is to compound them.
For example, LSTnet~\cite{lai2018modeling} integrates CNN, RNN, and autoregressive~\cite{yule1927method} techniques to extract both short-term and long-term patterns.
Using the occupancy rate of a freeway as an example~\cite{lai2018modeling} to explain these two patterns, the ``short-term'' patterns refer to the morning peaks against evening peaks, while the ``long-term'' patterns refer to the workday patterns against weekend patterns.
Clearly, a good forecaster needs to capture and distinguish both kinds of patterns.
Despite the superior performance deep learning methods have achieved, they tend to be overly complex, opaque, and incur high computational costs compared to traditional techniques.